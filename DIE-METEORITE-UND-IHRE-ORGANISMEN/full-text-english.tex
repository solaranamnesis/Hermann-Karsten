\documentclass[a4paper, 12pt, oneside]{article}
\usepackage[utf8]{inputenc}
\usepackage{fouriernc}
\usepackage{booktabs}
\setlength{\emergencystretch}{15pt}
\usepackage{fancyhdr}
\usepackage{graphicx}
\graphicspath{ {./} }
\begin{document}
\begin{titlepage} % Suppresses headers and footers on the title page
	\centering % Centre everything on the title page
	\scshape % Use small caps for all text on the title page

	%------------------------------------------------
	%	Title
	%------------------------------------------------
	
	\rule{\textwidth}{1.6pt}\vspace*{-\baselineskip}\vspace*{2pt} % Thick horizontal rule
	\rule{\textwidth}{0.4pt} % Thin horizontal rule
	
	\vspace{1.5\baselineskip} % Whitespace above the title
	
	{\LARGE The Meteorite and its Organisms}
	
	\vspace{1\baselineskip} % Whitespace above the title

	\rule{\textwidth}{0.4pt}\vspace*{-\baselineskip}\vspace{3.2pt} % Thin horizontal rule
	\rule{\textwidth}{1.6pt} % Thick horizontal rule
	
	\vspace{1\baselineskip} % Whitespace after the title block
	
	%------------------------------------------------
	%	Subtitle
	%------------------------------------------------
	
	{Professor Dr. Hermann Karsten} % Subtitle or further description
	
	\vspace*{1\baselineskip} % Whitespace under the subtitle
	
    {\small In Schaffhausen, with Figures\\ (Separate print from the journal \emph{Nature}, 1881, No. 14, 15, 16.)} % Subtitle or further description
    
	%------------------------------------------------
	%	Editor(s)
	%------------------------------------------------
    \vspace*{\fill}

	\vspace{1\baselineskip}

	{\small\scshape Halle [Prussia 1881]\\ Gebauer-Schwetschke'sche Book Publisher}
			
    Internet Archive Online Edition  % Publication year
	
	{\small Attribution NonCommercial ShareAlike 4.0 International } % Publisher
\end{titlepage}
\setlength{\parskip}{1mm plus1mm minus1mm}
\clearpage
\tableofcontents
\clearpage
\section{The Meteorite and its Organisms}
\paragraph{}
Of all natural phenomena, which has not only been more persistently admired and widespread but also dreaded, than the sporadically occurring meteors: storms accompanied by thunder and flashes and the most silent and mysterious drifting comets and fireballs? Of all unusual phenomena striking each and every one of us, which has remained inexplicable until this time other than these comets and meteorites, with rare cases of the latter approaching the Earth as balls of fire, tumbling down with thunderous patter? These stones are then discovered as angular fragments, slightly smooth and covered with a thin dark crust; this crust appears to be produced by the melting of the inner, unaltered mass, brought about by the heating undergone by the stone from friction against the atmosphere, through which it pierces at high speed. The friction during their passage through the atmosphere makes the stones glowing and luminous. In their various sizes they fall to the Earth, from many cubic feet of material weighing over 1000 talents, to bean size and sometimes even observed in the form of sand.

Some time ago I reported in these pages about small glowing stones recently fallen on people or in their immediate vicinity which belonged to the stone class of meteorites: here near Schaffhausen a man was shot in an open field through the arm, under circumstances which pointed only to a meteorite projectile. The case observed in France last year, when a farmer saw a stone fall beside him in a field and sold it to a museum only to become involved in a lawsuit, can still be remembered. These items are relatively insignificant, although certainly interesting knowledge. Many other infinitely greater ones are enumerated in the annals of natural history. A rain of stones fell near Shahabad in Hindustan in 1810, killing people and inflaming buildings. On the night of September 4, 1511 hundreds of stones fell in northern Italy; heavy pieces were brought to Milan by peasants; a monk lost his life due to this rain of stones and animals were killed in great numbers. Even the annals of the Chinese have reported, for centuries before our era, many cases of luminous meteors that fell to Earth. In 616 BCE, according to them, a fireball appeared in the sky from which stones fell to Earth after an explosion, killing ten people and smashing a wagon. Similarly, Greek and Roman writers mention the stone rain. Even the Christian Middle Ages, which was only concerned with the Creator and his family, not with the Creation, did not leave these strange manifestations of heaven completely ignored. Numerous observations of meteorites descending to Earth were recorded in modern times; Kesselmeyer, in his treatise on the origin of meteorite cases given to the Senckenberg Society in 1860, lists 647 meteoritic iron and stone falls with greater or lesser reliability. Many stones whose falls were observed in the glowing state have been collected, examined, and preserved; rocks that were sometimes identified as metals, sometimes mixtures of metals, and even coal and other organic elements.

However, the actual nature and historical development of these bodies, their origin, and their relationship to the Earth and the other bodies of the universe has remained shrouded in a seemingly impenetrable darkness.

The French physicist [Jean-André] Deluc made the first attempt to find an explanation for the fact of the falling ``fireballs'' of Earth ``sent by the Gods, [?] Drakel giving forth Batylien'', aerolites, meteorological, or aerial stones. He tried to prove that they were ejections from the volcanoes of Earth because, as a matter of fact, some of the compositions of many meteorites coincide with that of numerous volcanic rocks and outflows, or is at least very similar to them. This attempt failed because of the lack of enough ejecting power in our volcanoes, which was soon proven by calculation, and as there are such enormous meteorite masses found on the Earth's surface. In the state of Oregon, North America, below 40$^{\circ}$35$^\prime$ on the Pacific Ocean, there is an iron meteorite block whose part projecting above ground was estimated by [John] Evans, who took a piece of it, at 10,000 kilos. The most famous block of meteoritic iron was brought by [Peter Simon] Pallas traveling from Siberia --- famous because it prompted [Ernst] Chladni to pronounce the current theory of the nature of the meteorites --- weighing 688 kilos --- [Karl Ludwig von] Reichenbach estimates the annual weight of falling rock masses to be 4,500 Zentner.

The idea expressed by [Heinrich Wilhelm Matthias] Olbers in 1795, that these meteorolites are not ejecta of Earth's volcanoes, but those of the Moon, an idea which [Pierre-Simon] Laplace considered acceptable and was confirmed by many mathematicians through calculation, since the possibility was not contradictory: nevertheless, it gave way after considering all the necessary and favorable combinations in the positions of the Earth and Moon so that a single meteorite with incoming speed of about 2,300 meters per second would reach the Earth far too rarely to explain the numerous meteorites.

Likewise, the opinion expressed by other researchers that meteorites are products of the atmosphere or congregations of atmospheric origin derived from the Earth's surface could not be reconciled with the great distances, up to forty miles calculated for some fireballs, from which the meteorites fall to the Earth, and the extraordinary thinning of the atmosphere at an altitude of only ten miles, where solid bodies could not stay in place to accumulate up to masses as heavy as those which fall down to the Earth.

It remained, therefore, as the most acceptable hypothesis of those remaining, when in 1819 Chladni denied these luminous meteors and glowing meteor stones falling to Earth their meteorological nature and declared them to be cosmic bodies, with the stars, likely fragments of a shattered larger planet or independent planetary bodies whose orbits approach the Earth's orbit and in their relative smallness follow the attraction of the Earth itself. This idea probably lead to the discoveries, in that period from 1801 to 1807, of the four small planets orbiting in the middle point between Mars and Jupiter, by [Giuseppe] Piazzi, Olbers, and [Karl Ludwig] Harding, who also maintained that these were the shattered remains of a larger planet.

Yet Chladni suspected a connection between the meteorites and shooting stars and the comets; an idea that, like most new ideas, met with fierce opposition but after fifty years of strong support it seems to be confirmed, as found in the calculations of the orbits of some swarms of shooting stars by [Giovanni] Schiaparelli.

Throughout the year shooting stars are seen only as isolated, rapidly moving points of light, which cut through parallel paths of the fixed stars passing steadily and monotonously through the sky, however at certain times they appear to a surprised eye in great numbers, in whole swarms. The dense swarm appearing on November 12$^{th}$, according to H. A. [Hubert Anson] Newton's investigations, returns, at periods of 33 years, most brilliantly and numerous, appearing almost like a shower of light sparks to astonished terrestrial dwellers.

Less numerous, although more constant in its annual return and referred to in legend as the ``fiery tears of salvation (Laurentius)'', is the maelstrom developing on the 10$^{th}$ of August in the constellation of Perseus. Compare this Perseid Swarm against the November Swarm, which pours forth from the leonine constellation and is called the Leonid Swarm by astronomers. The nights of April 18--20, June 26--30, and December 9--11 are also characterized by a high frequency of shooting stars.

Schiaparelli has recently made the brilliant discovery that the orbits of certain comets coincides with those of the designated shooting star swarms; a perception that was soon confirmed by other astronomers and which is highly unfavorable to Chladni's hypothesis about the cosmic nature of the meteorites. For it is arguably not possible that small luminous bodies, which appear to us as shooting stars on the designated days, belong to the tail of a comet passing through or coming near to the Earth's orbit, and it seems reasonably possible that the individual parts of this comet tail, diverted from their orbit and following the Earth's gravity, are able to reach the Earth as meteorites passing through as balls of light, like Chladni suspected.

Before the invention of the telescope by [Galileo] Galilei only the largest comets entered the knowledge of man. Even today, most are not seen by people because of their distance from Earth or unfavorable observing times for astronomers. More recently, there have been so many comets discovered with the high-powered telescopes that one can assume their number is many thousands and that Kepler was right in saying that the number of comets in space is greater than the number of fish in the sea. Perhaps every day one or more comets approach the Earth so close that parts of their often twenty-million-mile-long tail appear to us at night as the sporadic shooting stars. Even so, meteorites continously fall to Earth, although only very few are seen and noticed by civilized man and so do not become public knowledge.

Based on the results from the latest astronomical research, the meteorites are pieces of foreign celestial bodies, indeed parts of a comet, and a study of their nature would therefore provide us a most excellent means for discovering the composition of the mass of these celestial bodies. This study, carried out with all the available means of modern chemistry, has revealed, as indicated above, that these meteorolites are composed from the same substances as our Earth.

Astronomical research on the physical properties of comets indicates that they are, so to speak, celestial bodies in the process of consolidation; that they consist of a glowing liquid or vaporous core and a frozen shell, a mantle, which is less hard, and corpuscles far from each other's vicinity which form a long luminous tail: corpuscles that are often seen as shooting star swarms on Earth after the main body of the comet has long since passed. The distance between the corpuscles forming the tail would have to be very considerable, since even the smallest stars can be seen shimmering without loss of light through the mass forming the tail, a length of more than 20,000 miles. At their extraordinary distance from the core of the comet these laggards probably follow gravity and fall down to Earth as meteorites.

Microscopic research discovered in these stones a mixture of granular crystalline metal and mineral bodies, above all iron in conjunction and mixed with nickel, cobalt, titanium, copper, tin, silica, magnesium and other substances. Some aerolites consist almost entirely of metallic iron and its metal alloys, while others almost exclusively of non-metallic mineral bodies. Depending on whether the iron alloys form the main mass, more or less coherently, or are in grains consisting of a mixture of quartz and silica compounds (very often as bronzite, olivine, and augite), or with the latter appearing more or less uniformly mixed with meteoritic iron grains, they become pallasites or mesosiderites. A third class, the most frequent of the falling meteor stones, consists of a lighter or darker matrix that is formed from a mixture of meteoritic iron, pyrrhotite, chromium, titanite, olivine, augite, bronzite, anorthite, quartz, etc., in which mass is found numerous small or large light-colored spherical or pear-shaped globules, $\chi$o$\nu\delta\rho$o$\iota$ [chondroi], apparently crystal druses of silica compounds stated as bronzite or enstatite. These mineralogically difficult-to-characterize, chemically very variable stones are called chondrites. Occasionally, these chondrites are completely black and in them are observed amorphous coal and bituminous substances that are probably decomposition products of organic compounds, about whose nature no conjecture could be made.

These chondrites, with their manifold undefinable inclusions, are now not merely conjectures; results from the most laborious research are contained in an epoch-making work: \emph{The Meteorite (Chondrite) and its Organisms} by Dr. Otto Hahn, which recently left the Laupp'sche press in Tübingen, and places the view on the nature of the meteorites in a completely new and unexpected light.

Many of my readers will remember the notice about \emph{Primordial Cell} published by the same author in 1879, i.e. about the simple organized bodies discovered in crystalline rocks. Who has read this book and not, regardless of his numerous depictions of the plants seen in the bedrock layers, entertained certain doubts! Even in meteorites, organisms and plant formations ought to be recognizable. Plants, one of which, akin to the algae and ferns, was described as \emph{Urania guilielmi} in honor of the German Emperor and depicted in the seventeenth table.

Notwithstanding some opposition against his discovery, the author of both these treatises, conscious of his good cause, has not been discouraged from further pursuing his discovery. Hundreds of thin sections had to be made, scrutinized and compared to each other in order to confirm the prior result and then to expand it: that some meteorites --- indeed, in the available work Hahn mentions eighteen distinct ones from the chondrite set of meteorites whose fall times are well-known --- consist almost entirely of a mixture of organisms. So, it is the microscope, which, as predicted by [Friedrich August von] Quenstedt (\emph{Handbook of Mineralogy} p. 722), has solved the enigma of the composition of the meteorites.

Hahn makes out from his descriptions of the organisms, which he found in these eighteen meteorites originating from various regions of the Earth, such classes as sponges, needle sponges, corals, and crinoids; he arrives at the result that the supposed enstatite and bronzite globules are nothing other than organisms, and this tissue, equivalent to corals, crinoids, shell gastropods, mollusks, etc. combined with inorganic substances to the utmost, so to speak, is microscopic silica and lime coral colonies, sponges, etc., whose globules form the main mass of the rock. Hahn claims that both individuals of one and the same organic type in these chondrites consist of various mineral substances, sometimes similar to the composition of enstatite, while in others that of bronzite: and vice versa, that one and the same mineral substance occurring in the organisms of different meteorites was assimilated and used to build up their bodies that served them.

Incidentally, the thirst of the vegetation center, the apparent ``crystallization center'' in these globules always lays eccentrically, a property that, as a distinguishing feature, does not give weight to their being crystal druses. For even in crystal druses the beginning of crystallization is often eccentric and quite on the edge, when the druses settle on a solid body very early, and a little less eccentric if this setting took place later; quite concentric if the beginning crystallization of the druses formed while buoyant in a liquid, as often occurs in organic substances, which is why oolite spheres are considered to be formed in a spring, a mineral water. However, the discovery of organisms in the chondrites, since held as glasses (!!) or crystallization processes, is correct and remains undoubtedly true for any who, with the requisite knowledge, engage in the investigation of these aerolites.

An excellent, highly accurate physical description of these chondrites is given by [Karl Wilhelm von] Gümbel in his instructive essay: ``About the Stone Meteorites Found in Bavaria'' (\emph{Minutes of the Mathematical-Physical Class of the Royal Bavarian Academy of Sciences in Munich}, 1878), from which some sentences may be quoted here to mark the position that science has currently taken on this issue.

``If one examines the results of the investigation of this, albeit limited, group of stone meteorites, then the perception that comes to the fore is that, in spite of some differences in the nature of their conglomeration, they are nevertheless governed by completely identical structural relations. All are undoubtedly debris, composed of small and large mineral grains, from the well-known roundish chondrules: which are usually completely preserved, but often appear as broken pieces, to the globs of metallic meteoritic substances, sulfur-iron, and chromium-iron. All these fragments are glued together, not cemented by an intermediate substance or a binder, as there are no amorphous, glassy, or lava admixtures at all. Only the fusion crust and black constrictions, which often appear on clefts and are similar to the crust, consist of amorphous glass, which, however, originated after falling within our atmosphere. In this melted crust, the denser melt-able and larger mineral grains are usually still embedded un-melted. The mineral splinters do not bear any traces of rounding or tumbling, they are sharp-edged and pointed. As for the chondrules, their surface is not smooth, as it would have been if they were the product of tumbling, rather it is always uneven, mulberry-like and warty, or multifaceted with a projection of crystalline surfaces. Many of them are elongated with a distinct tapering or sharpening in one direction, as is the case with hailstones. Often you encounter pieces which apparently must be regarded as parts of shattered chondrules. As an exception are twin-like connected beads, most common in those which meteoritic iron beads have grown. In numerous thin sections they are composed differently. Most often there is an eccentric, radiating fibrous structure which spreads from a point far from the center after tapering or slightly tattered lines spread like rays toward the outside. Since cuts made at various angles always reveal a columnar or needle-shaped arrangement, never leaves or lamellas in the substance forming these tufts, it seems to be columnar fibers from which such chondrules are built. With certain cuts, according to this assumption, in the cross-sections of the fibers that are perpendicular to the length direction, only irregularly angular minute fields are observed, as if the whole was composed of small polyhedral granules. Sometimes they appear as if there were several systems radiating in different directions in a sphere, as if the point of radiation were altered during its formation, so that a constant and seemingly confused elongated structure emerges. Towards the outside, against which the junction point of the radiating bundle is shifted unilaterally, the fiber structure normally becomes indistinct or replaced by a more granular aggregate formation. In none of the numerous ground-up chondrules could I observe that the tufts ran directly to the edge, as if the point of emission were outside the sphere, provided that it was completely preserved and not a mere shattered piece. The delicate transversely dividing fibers usually do not run along the entire length of the tuft, but rather they gradually sharpen, branch or end to allow others to take their place, so that in the cross-sections, a manifold, mesh-like or netted image is created. These fibrils consist, as has often been described above, of a mostly lighter core with a darker envelope that is dissolved by acids, while the latter resists. Highly curious are the bowl-shaped constructions, which seem to be meteoritic iron, that are generally only spread over a small part of the globules. The same unilateral striations, visible on the average as crescent bowed streaks, also appear inside the chondrules and provide strong evidence contrary to their being formed by a tumbling of some material, the entire arrangement of the tufted structure speaks to a resolution against their origin by tumbling. However, not all chondrules are the eccentric fibrous type; many, especially the smaller ones, have a fine-grained composition, as if they are composed of a mass of aggregated dust. Here too, the one-sided formation of the spheres is sometimes noticeable by an intensely greater compression of the dust pieces. --- The most common type of stony meteorite is predominately that of the so-called chondrites, the composition and structure of which coincide so much that we do not see how a common origin and the initial cohesion of these chondrites --- if not all meteorites --- could be in doubt.''

``The fact is that they enter our atmosphere as highly irregular pieces --- apart from the shattering within into several fragments, which is common, but cannot be assumed in all cases, especially if, by direct observation the falling of only a single piece is confirmed; it can be further concluded that they make their orbits in the heavenly space as demolished pieces of a single larger celestial body and in their absent-mindedness occasionally fall to Earth when they enter into the region of Earth's attraction. The lack of original lava-like amorphous constituents in connection with the external irregular form is likely to exclude from the geo- or cosmological points of view the assumption that these meteorites are ejections of lunar volcanoes, as is often claimed. --- Therefore, the meteorites appear to be a kind of first process of encasing the celestial bodies, but since they contain metallic iron --- to have been produced in the absence of oxygen and water.''

Our author fully agrees with this judgment on the aggregate form of the meteorites, but with the reservation that, as I have said, those small spherical pear-shaped bodies, which are the main constituents of the stone meteorolites, are not individual minerals, but exclusively organized ones, as well as almost the entire ripped and cracked silica matrix. In contrast to the meteorites described by Gümbel, in Knyahinya there is a slight shattered silica intermediate substance. ``All Life'' is a primeval forest, or rather, a small-scale polyp and sponge forest, a chaos of forms grown on one another, almost oddly like present day, only everything infinitely smaller.

On thirty-two photographic plates, 142 figures depict a myriad of discovered organisms, amongst others of earthly creation, which were used for comparison. Unfortunately, our author has been tempted by a critical detractor to abandon his method of self-drawing as done in \emph{Primordial Cell} and to present only photographs for explanation and authentication, instead of his own drawings; both side by side would have satisfied the reader more! For as natural as photographic images depict a particular state, a certain area, which is precisely in the focus of the microscope, and if light and color conditions are favorable, they are insufficient at providing the observer an idea of why a particular examined object maintains a certain characteristic, for a perspective drawing in which he could recognize such could be made by varying the focus (the visual range).

The drawing of a longitudinally intersected, druse-like globule was made by me with the help of an artist experienced and skilled in the depiction of natural history, especially microscopic objects, the Professor [Friedrich Eduard (?)] Metzger himself. After the most careful consideration, we have that which is truly peculiar to random objects, i.e. we sought the outwardly adherent ones from semblances caused by the refraction of light; it was initially obtained while proving that the object was organized. I believe that we have succeeded better and more fully than the photographer, so perfect are his pictures in accordance with the state of the photographic technique, in the various specimens of this organism in Table 1, 8, 9, 10, 11. Because of the delicacy in grinding, the partly foreign material covering the top of the object and the additional cracks which I thought to have originated by chance from the operation of sawing and grinding were not drawn in order to avoid overloading the complicated, greatly enlarged, yet meticulous picture with trivial things. Perhaps structural relations that could have served to provide counterevidence for the object being an organized body have been omitted out of too great a caution, for example, here and there a transverse partition in the branching fiber; but we considered them to be equivalent to the other concurrent lines that seemed to us to be random cracks. In a word, the picture gives what I want to show the reader as being observed by me as the organism, it is intended to replace a long, difficult-to-understand description.

This illustrated body comes from the meteorite fall of Knyahinya in Hungary on June 9, 1866, which in some parts, that is, in a twenty-seven pound piece, was reported as still lukewarm by the observer of the event, and the same having a penetrating garlic (selenium?) smell lasting three days. The stone came with rolling thunder out of a cloud as a glowing ball with a long tail, from which smaller ones came out on all sides. A large block weighing five-and-a-half Zentner at the same time penetrated 11$^\prime$ deep into the ground of a meadow.

This organism has been designated by Hahn as a coral; it is very similar to the \emph{Favosites} found in the oldest Silurian strata of the Earth's crust, as [Georg August] Goldfuss depicts these corals in his Tables 26 and 27; as well as the Silurian \emph{Calamopora} drawn by [Georg Amadeus Carl Friedrich] Naumann in the first table of his handbook. I chose this body, among the countless fragments of tissues --- which in their large-cell structure are easily identifiable as plant tissue ---, to represent them because it forms one of the chondrite globules to which the mineralogists have given special attention; globules that chemical analysis proves to be a kind of bronzite (enstatite), and which, because of their crystal druse form and columnar structure, resemble a crystalline body more than all else. The drawn individual is an approximately medium length section of one of these pear-shaped bodies; the upper and lower parts have been ground away, the edges are partly permeated by the iron silicates of the matrix; moreover, the whole organism is thoroughly transformed by a silicification of enstatite and the mentioned silica compounds. It consists of nearly straight, slightly radial tubes, somewhat widened towards the peripheral end, which sometimes, as in Figure 2, reveal a branching, as it seems, without partitions, at least in its younger parts; perhaps in the lower, narrow end with partitions at right angles to the longitudinal walls. Individual parts of this tube system, approximately midway between the nearly parallel ones, are slightly bent and appear to end in a thinned and rounded tip. All the tubes are, as it seems to me and shown in Section b of Figure 1, filled with a series of spherical cells with thick walls that lie directly adjacent to each other in the older parts, while in the younger parts the tube membrane seems to be proportionately thicker, probably elongated cavities, a bore of the tube, and can be seen as small dark edged vesicles which lie at regular intervals, as shown in Section a of Figure 1. The transitional forms between these two parts of the tubes I was not able to exactly recognize. Between the tubes there is a cloudy dark yellowish-brown to brown mass, in which a series of light vesicles can be seen; perhaps they are the vesicles of the contents lying above, for the most part ground away. As I said, Hahn designated this body as \emph{Favosites} by maintaining these apparent vesicles as intersecting channels, the so-called bud channels. In fact, it has, apart from its extraordinary smallness, the greatest resemblance to the images of the above-mentioned corals; I hold the same view, based on one specimen, for a colorless thread alga, for a hysterophyme, that is, for \emph{Leptomitus} or \emph{Leptothrix}; without sufficient material, as only Hahn himself commands today and which has been used in the most diligent way, it would be too daring an enterprise to set up a position different from his own.

In any case, this body is not a druse of needle-shaped or columnar crystals, as the mineralogists think, but an organized entity; for real crystals that precipitate out of evaporating or cooling solutions are structureless and homogeneous.

Of great interest to elucidating the nature of these organism of the meteorites are the highly similar structures recently discovered by Paul. F Reinsch in coal; a discovery that the gentleman editor had the kindness to bring to my knowledge.

According to Reinsch's observations, individual layers of Saxon coal consist of 20\% of such organisms, just as the chondrites are mostly composed of them. The plants discovered by Reinsch are very small, microscopic structures, and they too occur in a few forms, but in the greatest number together forming the basis of the coal seams referred to; in some cases they consist, similar to the organism drawn in Figures 1 and 2, of branched out concentric fibers, more or less free cells. Reinsch considers them to be algae and fungi, such as slime molds, and that he too, based on valid reasons, expressly protests against their inorganic nature. Also, these coal organisms agree with those of the meteorite, in that their shared ancestors (in the pyrites) are mineralized or silicified. I also consider these organizations of hard coal to be hysterophyms of decaying and rotting plants composing the coal: hysterophyms whose nature and development I repeatedly highlight in my recent \emph{German Medical Flora} (1880); organizations that any impartial and careful observer can see, in the mentioned manner, as plant and animal tissue cells, as well as the metamorphoses that develop in them. In the case discovered by Reinsch the necrobiotic metamorphosis occurs underwater, and those discovered by Hahn in an atmosphere with varying degrees of moisture; in both cases they are the simple forms of cell reproduction as taught in the study of contagions and miasmas and how I present them in my \emph{Decay and Contagion} (1872).

Hahn further found that all the stone meteorites he examined, and only about these does he express himself in the available work, contain the same organized creatures. A result that had already been obtained from the mineralogical investigation, with respect to their chemical-physical properties; and this fact leads him on p. 44 to the conclusion that: ``all these chondrites are debris that orbited after the destruction of the planet until, fortunately, they came into the attraction of the Earth.''

The forms of the creatures so far recognized in the chondrites are all associated with water; the whole mass of these meteorites seems to have been built underwater, the countless microscopic organisms either petrified retroactively or, more likely based on the chemical analysis of these bodies, combined in their own way with the mineral substances dissolved in this water and assimilated the same, similar to how present-day mussels, corals, bacillaria, equiseten, and various Vibrionaceae skins silicify and calcify in a similar manner to the bones of vertebrates. Ultimately, they were cemented together by the dried-up reside of the silica rich nutrient liquid into a coherent silica rock mass. One also sees, therefore, countless small translucent and transparent organizations --- at least in the Knyahinya meteorite --- heaped one upon another, and this makes it very difficult to recognize the actual form of most of them, since their presence, even to those who are familiar with microscopic organic forms, is difficult to perceive, especially being unfamiliar forms.

The individually organized globules and tissue fragments are interim-stored in the silica mass, as I said, and in it there are found large and small scattered splinters of metallic iron and nickel, and titanium or chrome-iron compounds, some of which seem to merge with the silica mass and also, in some cases, to partially saturate the organisms, however the metallic iron alloys are present as sharp-edged and irregularly angular forms. The manner of development of these metallic iron splinters, when considering the vegetative activity of the organisms, as Hahn naturally does, and based on experiments and observations I have made in this direction, may be twofold: either the metal may be the secretions from some kind of dissolution of siliceous, chloric, chrome, etc. with reduced and metallic iron existing as precipitates, as happens with silver and mercury salts by fungal vegetation; or, like clay and the Alkalies, like natron, potash, lime, magnesia, etc. is absorbed by the assimilating cell membrane and used in the actual development of its constitution\footnote{A detailed account of the assimilating and organizing activity of the living cell membrane was given recently (1880) in my \emph{Botany}, pp. 17-22.}, as this membrane continuously forms more and greater alkaline compounds until finally its original organic elements are altogether expelled, so that, like magnesia or lime salts, only metallic alloys are left remaining. The organisms of this last-world [Letztwelt?] only provide us with the first developmental stages of these metal compounds as evidence for this theory, as considered by Hahn and laid down in my treatise \emph{Chemistry of the Plant Cell}. The organisms of the meteorites, however, based on the extraordinary smallness in which they most often occur, may indicate physical conditions different from the various ones of today, perhaps considerably hotter or cooler temperatures, etc. As to what happens under such unfamiliar conditions to inorganic elements assimilated by cell membranes, that remains completely unknown to us. The fact that organisms continue to grow and multiply at high temperatures, for instance at the boiling point of water, albeit in a much smaller form, I mention in the referred to treatise \emph{Chemistry of the Plant Cell}. Since then, I have convinced myself that even at higher temperatures, i.e. at 150$^{\circ}$, the vitality of plant organization does not disappear completely, but rather the content of individual tissue cells can still develop, even if sparsely, but usually as tender and small forms. On the other hand, organisms also continue to multiply at low temperatures below freezing and also with significantly smaller sizes than at positive 30 to 35$^{\circ}$ C. That bacteria can be kept alive for one hour at a temperature of negative 100$^{\circ}$ C was repeatedly observed; if the experiment could be continued long enough, then one would perhaps find this scale-down law confirmed.

In any case the present book by Hahn, with the brilliant discovery of a new world of organisms brought to Earth in the meteorites, calls upon us to revise many tenets which had already appeared to be certain results of observation and calculation. If we realize that the supposition, that meteorites are parts of comets, is correct then comets cannot be incandescent molten bodies that are only cold on the exterior and then broken into individual fragments; for the stone meteors are not heated to significant degrees of temperature before they meet our atmosphere, as they would have melted into a glass! Instead there is only a slight influence of heat --- perhaps, as previously implied, from the frictional heat against the atmospheric air during its entry --- on the outer surface as a uniformly thick crust around each of the fallen stones. It seems this fusion crust is formed for the most part only after the commonly observed, and heard, bursting of the entire mass forming the luminous orb: for every single angular piece thus formed is wrapped all around with an, as it appears, equally thick fusion crust; it therefore only came into existence in the lower and denser regions of the atmosphere. But if these meteorites were originally part of a comet, then it is not in a molten, fiery-liquid state; its light is acquired, i.e. reflected; and its mass is of such a nature that it was neither heated to melting nor rose to a level that would make the life of organisms impossible. It would correspond to the idea of Hahn's and the Neptunists about the origin of our Earth as not being from a fiery-liquid, but an aqueous-liquid, and its little bit of fragmented crust as cooled by evaporation. For probably ``the first beginning of our planet, and therefore of all planets, was an organic formation (p. 40), --- the cell, it is maintained so long as light rays hit the Earth! (p. 50)''.

But regarding the already touched upon idea of the terrestrial origin of the meteorites, I would like to again bring to mind the historically witnessed fireballs and meteorolites; would not these meteorolites be melted down to glass in their fall if these bodies first came into being in the atmosphere only as trade-wind dust?

According to Hahn's view, the whole solid mass of the known celestial bodies is the product of organized activity; according to Hahn, cells form from the chaos of elements, which in addition to the so-called organic elements (carbon, oxygen, hydrogen, nitrogen) also contain great amounts of inorganic elements, i.e. clays and metals, by assimilating and incorporating them into their own mass. This energetic vegetation process of the organism, spread through the entire vaporous and liquid mass of the forming celestial bodies, might also be the emissary of light production, similar to what we know of some luminous animals, plants, and hysterophytes (fission fungi) of our Earth, and that these light generating organisms would therefore gleam stronger where they are found together in great numbers.

The fact that these meteorites, permeated with organized bodies, did not undergo any melting temperatures before encountering our atmosphere is undoubtedly demonstrated by their structure as revealed in the microscope. Therefore, they entered our atmosphere in an un-melted, cold condition; formed in an another unknown distant place, they are now available to us.

Perhaps even the cosmic origin idea, at least for this type of meteorite, must be abandoned in favor of their formation as conglomerates of meteor dust or trade-wind dust of similar material, as [Pieter van] Musschenbroek, Dominic Tata, [Eugène Louis Melchior] Patrin, [Ernst Friedrich] Wrede, Egen, von Hof, Kesselmeyer and others would maintain, although the development of such a conglomerate with today's physical knowledge and experience cannot be understood in detail.

These above-mentioned authors, Kesselmeyer quite superbly, consider the fireballs and falling meteorites as atmospheric sublimation structures of mineral fumes emitted by our volcanoes; and, admittedly, the chemist analyzing the volatility of all these mineral substances is at a great disadvantage in his quantitative analysis before this property of solid bodies is adequately discerned to exist, often only made perceptible in a regrettable way.

Furthermore, any visitor of an active volcano knows the interesting phenomena of the continuous steam of these volcanoes, often glowing at night-time. With water at the same time, which constitutes the greater part of this vapor welling up the steady crater, there is pulverized or vaporous elements of rocks that are pervaded by a blistering mineral water steam: pulverized masses, so-called volcanic ash, which during high activity add molten rock to the more or less comprehensive rock fragments. The latter soon fall back to Earth, but the pulverized portion is carried along with the water vapor to astonishing heights, dispersing in the upper regions of the atmosphere. With great pleasure I viewed this fascinating spectacle, which was granted to me by Puracé in the Cordilleras, a 5000$^{\prime}$ high column of vapor, which in the calm atmosphere swelled vertically in height, at first tempestuously swirling out of the crater's summit, then rising more slowly, until, at a specific height, it spreads out horizontally and forms a cloud layer, this in turn again provokes the upper fringes of the atmospheric layers. All the while, dust particles from the surface of the ground swirl vertically upwards in height, also larger light bodies, dry foliage, butterfly wings, etc., themselves carried to altitudes where they vanish from sight, witnessed especially in the hot lowlands of the equatorial region at the time of the turn of the year, when light little clouds form here and there, whose shadows thrown on the heated dry soil of the burned Llanos cause a slight cooling in some places sufficient to cause the emergence of burgeoning air vortices, that with the clouds tread along and sweep off the lightweight dust particles and carry them skyward until they disappear from the eye. How large masses accumulate in the upper regions of the atmosphere in this way, frequently sinking in often very remote regions, is a lesson that the above-mentioned phenomena of meteor- and trade-wind- dust teaches, the microscope proving the mixture to be of organized and unorganized bodies. That the still-viable organized parts of this dust, when it mixes with humid layers of air in the atmosphere can awaken its life expressions, its assimilating activity is capable of continuing just as it can be observed in the development of bacteria and their relatives and how they live in the humid chamber of the microscopist, is probably not in doubt; but how far the organizing processes of these microscopic cells can continue to be sustained in these amusingly frigid heights, we still have no idea; yet perhaps if such can be drawn from Hahn's surprising report, then the act of condensation of clouds impregnated with derivatives of trade-wind dust would not be that puzzling to us, but we doubt whether these phenomena can be associated.

That tremendous masses, which certainly originate in Earth's atmosphere, are capable of coagulating in this realm is demonstrated by ice masses that from time-to-time fall down to Earth. I myself observed a hailstorm one day in southern Bavaria whose grains were the size of hen's eggs, and these were not rounded like ordinary hailstones but sharp-edged pieces, which seemed to be fragments of larger masses; an occurrence also observed by [Captain] Delcross [Bibliothéque Universelle, Vol. 13, p. 154]. These sharp-edged chunks of ice strongly remind one of the bursting of the stone meteorites at perigee. In the year 1802, on May 28 at Puztemischel in Hungary, during a hailstorm a chunk of ice 3$^{\prime}$ in length, 3$^{\prime}$ in width, and 2$^{\prime}$ depth fell to the ground; its weight was estimated at 11 Zentner. [Christian Leopold von] Buch relates from [Benjamin] Heyne's \emph{Tracts Historical and Statistical on India} of an ice-mass that fell at Seringapatam in India that was the size of an elephant, so that despite the great heat of this country, it took a period of two days to melt. These ice-masses develop by the freezing of rain clouds that suddenly interact with cold and violent dry airflows. In such hailstones even metal cores were observed; as in Mayo, Ireland on June 21, 1821. Could the clashing of airflows impregnated with miscellaneous mineral gases and organisms in the highest regions of the atmosphere coagulate into the chondrite masses? On July 14, 1860 at Dharamsala in the Lahore area stones fell with an explosion, and although melted on the surface, were said to have been so cold that people who wanted to excavate them could not hold them in their hands because their fingers blistered from the coldness. Did these stones bring down the coldness of outer-space or the temperature of the Earth's upper atmosphere to these people? Being aware of the meteorites of Dharamsala, Thomas Carnalley recently sustained an ice-cylinder flank that was heated in vacuum up to positive 180$^{\circ}$ C.

The friction between such pulverized masses, as occurs in the trade-wind dust, undoubtedly generates electrical voltage and could cause it to come together, a coming together that in the presence of enough quantities of water vapor occurs without any actual melting.

That the implied friction against the atmosphere, of bodies reaching the Earth's atmosphere, is not alone sufficient to explain the glow and heating up of the meteorites, as was pointed out as early as 1835 by von Hof who brought to attention that they do not start in the highest and thinnest air layers and become extinguished in the lowest and densest, instead they steadily attain an ever-increasing fall velocity until reaching the Earth's surface.

The diversity of the shooting stars and fireballs indicates an extraordinary diversity of fall velocities of both meteors. While shooting stars rush through the sky at speeds of 10--20 miles per second, the much larger fireballs move only at a speed of one or a few miles per second. The same falling occurs for the iron meteorites, which sometimes arrive at the Earth's surface in a red-hot semi-liquid, molten state so that little rocks penetrate into them, for instance as was observed in 1808 with Parma [Borgo San Donino] and with Belaya Zerkara [Bjelaja Zerkov] in Russia. The stone meteorites have also been found in a semi-malleable state after their fall to the Earth, for example, near Cold Bokkeveld on the Cape of Good Hope where on October 13, 1838 a fireball, along with violent explosions, and many initially soft, black, carbonaceous, ammoniacal-fume-releasing stones permeated with water and bituminous substances fell with more than several hundred pounds weight still soft and only hardening later. A similar stone fell to Earth in 1864 at Orgueil; it was soft and could be crushed between the fingers; only the fusion crust and a cement of soluble salts held it together. Should phenomena of such different natures: fireballs that sometimes send semi-liquid molten metal masses, while at other times water-soaked clay conglomerates, to the Earth not perhaps owe their origin to entirely different processes? Fireballs and shooting stars possessing several origins?

There remains much to be observed; for the moment, in accordance with Hahn's procedures, all the meteorites should once again be thoroughly examined.

If this were the only result of Hahn's work, then the gratitude of science would be due for this suggestion; however, his merit, by discovering the organized nature of the greater part of the meteorites, is a positive one and I only wish that he actively proceeds down this path.
\clearpage
\section{Figures}
\clearpage
\pagestyle{fancy}
\fancyhf{}
\rhead{Figure 1}
\cfoot{\thepage}
\begin{figure}[t]
\includegraphics[width=\textwidth,height=\textheight,keepaspectratio]{fig1.jpeg}
\centering
\end{figure}
\clearpage
\rhead{Figure 2}
\begin{figure}[t]
\centering
\includegraphics[width=\textwidth,height=\textheight,keepaspectratio]{fig2.jpeg}
\end{figure}
\clearpage
\end{document}
