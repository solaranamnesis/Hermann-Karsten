\documentclass[a4paper, 11pt, oneside]{article}
\usepackage[utf8]{inputenc}
\usepackage[T1]{fontenc}
\usepackage[ngerman]{babel}
%\usepackage{fbb} %Derived from Cardo, provides a Bembo-like font family in otf and pfb format plus LaTeX font support files
\usepackage{yfonts}
\usepackage{booktabs}
\setlength{\emergencystretch}{15pt}
\usepackage{fancyhdr}
\usepackage{graphicx}
\usepackage{microtype}
\graphicspath{ {./} }
\begin{document}
\frakfamily
\begin{titlepage} % Suppresses headers and footers on the title page
	\centering % Centre everything on the title page
	%\scshape % Use small caps for all text on the title page

	%------------------------------------------------
	%	Title
	%------------------------------------------------
	
	\rule{\textwidth}{1.6pt}\vspace*{-\baselineskip}\vspace*{2pt} % Thick horizontal rule
	\rule{\textwidth}{0.4pt} % Thin horizontal rule
	
	\vspace{1.5\baselineskip} % Whitespace above the title
	
	{\scshape\LARGE Die Meteorite und ihre Organismen}
	
	\vspace{1\baselineskip} % Whitespace above the title

	\rule{\textwidth}{0.4pt}\vspace*{-\baselineskip}\vspace{3.2pt} % Thin horizontal rule
	\rule{\textwidth}{1.6pt} % Thick horizontal rule
	
	\vspace{1\baselineskip} % Whitespace after the title block
	
	%------------------------------------------------
	%	Subtitle
	%------------------------------------------------
	
	{\scshape Prof. Dr. Hermann Karsten} % Subtitle or further description
	
	\vspace*{1\baselineskip} % Whitespace under the subtitle
	
    {\scshape\small In Schaffhausen, mit Abbildungen\\ (Separatabdruck aus der Zeitschrift \emph{Die Natur}, Jahrgang 1881, Nr. 14, 15, 16.)} % Subtitle or further description
    
	%------------------------------------------------
	%	Editor(s)
	%------------------------------------------------
    \vspace*{\fill}

	\vspace{1\baselineskip}

	{\small\scshape Halle [Preu}\small "s{\small\scshape en 1881]}
	
	{\small\scshape{Gebauer-Schwetschke'sche Buchdruckerei}}
	
	\vspace{0.5\baselineskip} % Whitespace after the title block

    \scshape Internet Archive Online Edition  % Publication year
	
	{\scshape\small Namensnennung Nicht-kommerziell Weitergabe unter gleichen Bedingungen 4.0 International} % Publisher
\end{titlepage}
\setlength{\parskip}{1mm plus1mm minus1mm}
\clearpage
\tableofcontents
\clearpage
\section{\frakfamily{Die Meteorite und ihre Organismen}}
\paragraph{}
Welche von allen Naturerscheinungen ist von dem Menschen l"anger und h"aufiger nicht nur bewundert, sondern auch gef"urchtet worden, als die sporadisch austretenden Meteore: die von Blitz und Donner begleiteten Gewitter und die meistens lautlos und geheimnisvoll dahinziehenden Kometen und Leuchtkugeln? Welche von allen, einem Jeden auffallenden ungew"ohnlichen Erscheinungen ist bis auf unsere Tage unerkl"arlicher geblieben, als diese Kometen und Meteorite, welche Letztere in seltensten F"allen sich als Feuerkugeln der Erde n"ahern und selbst mit donnerndem Prasseln auf sie herabst"urzen? Man findet dann diese Steine als eckige etwas gegl"attete Bruchst"ucke, mit einer d"unnen dunklen Rinde bedeckt; wie es scheint, eine durch Schmelzung der inneren, unver"andert gebliebenen Masse erzeugte Rinde, durch die Erhitzung hervorgebracht, die der Stein durch die Reibung gegen die Atmosph"are erf"ahrt, die er in gr"o"ster Geschwindigkeit durchschneidet. Diese Reibung w"ahrend ihres Durchganges durch die Atmosph"are macht die Steine gl"uhend und leuchtend. Die in den verschiedensten Gr"o"sen auf die Erde herabgefallenen sind von vielen Kubikfu"s Inhalt und "uber 1000 Zentner schwer, bis zu Bohnengr"o"se, z. T. selbst in Form von Sand beobachtet worden.

Vor einiger Zeit gab ich Bericht in diesen Bl"attern "uber kleine, erst k"urzlich auf Menschen oder in deren unmittelbare N"ahe herabgefallene gl"uhende Steinchen, welche zur Klasse der Meteorsteine geh"orten: hier bei Schaffhausen wurde ein Mann auf freiem Felde durch den Arm geschossen, unter Umst"anden, die nur auf einen Meteorsteine als Geschoss deuteten. Der in Frankreich im vorigen Jahre beobachtete Fall, wo ein Bauer auf dem Felde einen Stein neben sich herabfallen sah, ihn einem Museum verkaufte und deshalb in einen Prozess verwickelt wurde, ist noch in Aller Erinnerung. Dies waren verh"altnism"a"sig unbedeutende, wenn auch wegen sicherer Kenntnis interessante Erscheinungen. Viele andere unendlich gro"sartigere werden in den Annalen der Naturgeschichte ausgez"ahlt. Ein 1810 bei Shahabad in Hindostan niedergefallener Steinregen t"otete Menschen und entz"undete Geb"aude. In der Nacht vom 4. Sept. 1511 fielen in Oberitalien Hunderte von Steinen; zentnerschwere St"ucke wurden von den Bauern nach Mailand gebracht; ein M"onch verlor durch diesen Steinregen das Leben, auch Tiere wurden in Menge get"otet. Schon die Jahrb"ucher der Chinesen berichten seit Jahrhunderten vor unserer Zeitrechnung "uber zahlreiche F"alle leuchtender Meteore, welche zur Erde fielen. Im Jahre 616 vor Christus erschien nach ihnen eine Feuerkugel am Himmel, aus der nach einer Explosion Steine zur Erde fielen, die 10 Menschen t"oteten und einen Wagen zerschmetterten. Ebenso erw"ahnen griechische und r"omische Schriftsteller der Steinregen. Selbst das christliche Mittelalter, das nur mit dem Sch"opfer und den Seinigen, nicht mit der Sch"opfung sich besch"aftigen durste, lie"s diese merkw"urdigen Himmelserscheinungen nicht ganz unbeachtet. Zahlreiche in der Neuzeit sich vermehrende Beobachtungen von Meteoriten, die sich zur Erde herabsenkten, wurden verzeichnet; Kesselmeyer f"uhrt in seiner der Senkenbergischen Gesellschaft 1860 "ubergebenen Abhandlung "uber den Ursprung der Meteorsteinf"alle 647 Meteor-Eisen und -Steinf"alle von mehr oder minderer Zuverl"assigkeit auf. Viele Steine, deren Herabfallen im gl"uhenden Zustande beobachtet worden war, wurden gesammelt, gepr"uft und aufbewahrt; Gesteine, die z. T. als Metalle, z. T. als Mischungen dieser mit Gesteinen, selbst mit Kohle und anderen organischen Elementen erkannt wurden.

Die eigentliche Natur und Entwicklungsgeschichte dieser K"orper, ihre Entstehung, ihr Verh"altnis zur Erde und zu anderen K"orpern des Weltalls blieb jedoch in ein, wie schien, undurchdringliches Dunkel geh"ullt.

Den ersten Versuch, f"ur die Tatsache der als "`Feuerkugeln"' zur Erde gefallenen "`von den G"ottern gesendeten, Drakel spendenden Batylien"', der Aërolithen, Meteor- oder Luftsteine eine Erkl"arung zu finden, unternahmen die franz"osischen Physiker [Jean-André] Deluc. Dieselben versuchten es, sie als Ausw"urflinge der Vulkane unserer Erde nachzuweisen, weil, wie in der Tat, dir Zusammensetzung mancher Meteorsteine mit derjenigen vieler vulkanischen Gesteine und Ausfl"usse "ubereinkomme, wenigstens denselben h"ochst "ahnlich sei. Dieser Erkl"arungsversuch scheiterte an der bald durch Rechnung nachgewiesenen unzureichenden Wurfkraft unserer Vulkane f"ur so gewaltige Meteorsteinmassen, wie sie sich auf der Erdoberfl"ache finden. Im Staate Oregon in Nordamerika liegt unter 40$^{\circ}$35$^{\prime}$ am Stillen Ozeane ein Meteoreisenblock, dessen "uber den Boden vorragender Teil von Dr. [John] Evans, der ein St"uckchen davon mitnahm, auf 10000 kilo gesch"atzt wurde. Der ber"uhmteste von [Peter Simon] Pallas aus Sibirien mitgebrachte Block von Meteoreisen --- ber"uhmt weil er [Ernst] Chladni veranlasste, die heute giltige Theorie "uber die Natur der Meteorsteine auszusprechen --- wog 688 kilo. --- [Karl Ludwig von] Reichenbach sch"atzt das Gewicht der j"ahrlich auf die Erde herabfallenden Meteorsteinmassen auf 4500 Ztr.

Auch die von [Heinrich Wilhelm Matthias] Olbers 1795 ge"au"serte Idee, diese Meteorolithe seien nicht Ausw"urflinge der Vulkane der Erde, sondern derjenigen der Mondes, eine Idee, die auch [Pierre-Simon] Laplace f"ur annehmbar hielt und die darauf von vielen Mathematikern durch Rechnung, als der M"oglichkeit nicht widerstreitend, best"atigt wurde: musste dennoch der "Uberlegung weichen, dass das Zusammentreffen aller der notwendigen, g"unstigen Kombinationen in der Stellung der Erde zum Monde, damit ein von diesem mit Anfangsgeschwindigkeit von etwa 2300 Mtr. in der Sekunde, emporgeschleuderter K"orper zur Erde gelange, viel zu selten eintreffe, um die zahlreichen Meteorsteine zu erkl"aren.

Auch die von anderen Forschern ausgesprochene Meinung, die Meteorsteine seien Erzeugnisse der Atmosph"are, oder Kongregate von Atmosph"arilien, die dem festen Erdk"orper entstammten, konnte nicht vereinigt werden mit der gro"sen, f"ur manche Leuchtkugeln bis auf 40 Meilen berechneten Entfernung, aus der die Meteorolithe auf die Erde herabst"urzen, und mit der au"serordentlichen Verd"unnung der Atmosph"are schon in einer H"ohe von 10 Meilen, wo feste K"orper unm"oglich sich schwebend erhalten und bis zu so schweren Massen ansammeln k"onnten, wie sie hin und wider auf die Erde herabfallen.

Es blieb daher, als die annehmbare Hypothese, nur die 1819 von Chladni aufgestellte "ubrig, diesen leuchtenden Meteoren und gl"uhend zur Erde kommenden Meteorsteinen eine meteorische Natur abzusprechen und sie, allen "ubrigen Gestirnen gleich, f"ur kosmische K"orper zu erkl"aren, und zwar f"ur wahrscheinliche Bruchst"ucke eines zertr"ummerten gr"o"seren Planeten, oder f"ur selbst"andige planetare K"orper, deren Bahnen sich der Erdbahn, oder der Erde selbst, so sehr n"aherten, dass die bei ihrer relativen Kleinheit der Anziehungskraft der Erde folgen und sich mit derselben vereinigen mussten. Auf diese Idee hatte wohl die Entdeckung der 4 kleinen Planeten zwischen Mars und Jupiter in jener Zeit von 1801 bis 1807 durch [Giuseppe] Piazzi, Olbers und [Karl Ludwig] Harding gef"uhrt, welche gleichfalls damals mit Olbers f"ur Zertr"ummerungsprodukte eines gr"o"seren, an ihrer Stelle fr"uher eine Mittelstra"se ihrer Bahnen wandelnden Planeten gehalten wurden.

Schon Chladni vermutete einen Zusammenhang von Meteoriten und Sternschnuppen mit den Kometen; eine Idee, die wie meistens neue Ideen auf heftigen Widerspruch stie"s, jedoch nach 50 Jahren eine Kr"aftige St"utze, wie es scheint eine Best"atigung, fand in den Berechnungen der Bahnen einiger Sternschnuppenschw"arme durch [Giovanni] Schiaparelli.

Zwar w"ahrend des ganzen Jahres sieht man Sternschnuppen als vereinzelte, rasch bewegliche Lichtpunkte die parallelen Bahnen der stetig und monoton am Firmament vor"uberziehenden Fixsterne durchschneiden, aber zu gewissen Zeiten erscheinen sie dem "uberraschten Auge in gr"o"serer Menge, in ganzen Schw"armen. Diese Zeiten bilden Perioden, die f"ur den am 12. November erscheinenden dichten Schwarm nach H. A. [Hubert Anson] Newton's Untersuchung eine L"ange von 33 Jahren umfassen, nach deren Verlaufe derselben am gl"anzendsten und zahlreichsten wiederkehrt, fast einem Lichtfunkenregen gleich sich den staunenden Erdbewohnern zeigt.

Weniger zahlreich, aber in seiner j"ahrlichen R"uckkehr gleichm"a"siger, erscheint am 10. August der, von der Legende als "`die feurigen Tr"anen des heil. Laurentius"' bezeichnete, aus dem Sternbild des Perseus sich entwickelnde sog. Perseidenschwarm, wogegen der Novemberschwarm, der dem Sternbild des L"owen entquillt, Schwarm der Leoniden von den Astronomen genannt wird. Auch die N"achte vom 18--20. April, vom 26--30. Juni, vom 9--11. Dezember sind durch gr"o"sere H"aufigkeit von Sternschnuppen ausgezeichnet.

Schiaparelli hat nun die gl"anzende Entdeckung gemacht, dass die Bahnen gewisser Kometen mit denen der bezeichneten Sternschnuppenschw"arme zusammenfallen; eine Wahrnehmung, die bald von anderen Astronomen best"atigt wurde und die der Chladni'schen Hypothese "uber die kosmische Natur der Meteorsteine h"ochst ung"unstig ist. Denn es ist wohl nicht wahrscheinlich, dass die kleinen leuchtenden K"orper, welche uns an den bezeichneten Tagen als Sternschnuppenregen erscheinen, dem Schweife des der Erdbahn nahe vor"ubergezogenen oder sie durchkreuzenden Kometen angeh"oren, und es schien demnach wohl annehmbar, dass einzelne den Erdk"orper nahe streifende Teile dieses Kometenschweif von ihrer Bahn abgelenkt, und der Erdanziehung folgend, als Meteorsteine gebende Leuchtkugeln auf die Erde gelangen k"onnen, so wie Chladni es vermutete.

Vor Erfindung des Teleskope durch [Galileo] Galilei kamen nur die gr"o"seren Kometen zur Kenntnis des Menschen, die sich der Erde einigerma"sen n"ahern. Auch noch heute werden die Meisten, wegen ihrer Entfernung von der Erde oder weil sie den beobachtenden Astronomen nicht zur g"unstigen Zeit erscheinen, nicht gesehen. In neuerer Zeit sind mit den lichtst"arkeren Teleskopen so zahlreiche Kometen entdeckt worden, dass man annehmen darf, ihre Zahl betrage viele Tausende und dass Kepler Recht hatte, wenn er sagte, die Zahl der Kometen im Weltraume sei gr"o"ser, als die Zahl der Fische im Meere. Vielleicht t"aglich n"ahert sich einer oder der andere der Kometen unserer Erde so weit, dass Teile seines oft 20 Millionen Meilen langen Schweife uns bei Nacht als sporadische Sternschnuppen erscheinen. Ebenso fallen wohl zu allen Zeiten Meteorsteine auf die Erde herab, von denen aber nur die allerwenigsten von zivilisierten Menschen gesehen und beachtet werden, daher nicht zur allgemeinen Kenntnis gelangen.

Die Meteorsteine w"aren demnach, auch nach den Ergebnissen der neuesten astronomischen Forschung, Teile fremder Himmelsk"orper, und zwar Teile irgend eines Kometen, und das Studium der Natur dieser Steine w"urde uns demnach das trefflichste Mittel geben, die Zusammensetzung der Masse jener Himmelsk"orper kennen zu lernen. Dieses Studium, welches mit allen, der heutigen Chemie zu Gebote stehenden Mitteln ausgef"uhrt wurde, hat nun, wie oben angedeutet, ergeben, dass diese Meteorolithe aus den gleichen Stoffen zusammengesetzt sind, wie unsere Erde.

Die Forschungen der Astronomie "uber die physikalischen Eigenschaften der Kometen schienen daraus hinzudeuten, dieselben seien gleichsam in der Konsolidation begriffene Himmelsk"orper, sie best"anden aus einem gl"uhend fl"ussigen oder dampff"ormigen Kerne und einer erstarrten H"ulle, einem Mantel, welcher in Form kleiner, fester, weit von einander entfernter K"orperchen den Umkreis und den langen leuchtenden Schweif derselben bilde: K"orperchen, die als Sternschnuppenschw"arme von der Erde aus oft noch gesehen werden, nachdem der Hauptk"orper des Kometen l"angst vor"uberzog. Die Entfernung der den Schweif bildenden K"orperchen von einander m"usste eine sehr betr"achtliche sein, da man durch eine solche die Dicke eines Kometenschweif bildende Masse von mehr als 20000 Meilen Ausdehnung hindurch, noch die kleinsten Sterne ohne Lichtverlust hindurchschimmern sieht. Bei ihrer au"serordentlichen Entfernung von dem Kerne des Kometen k"onnten diese Nachz"ugler dann wohl der Schwerkraft folgen und auf die Erde als Meteorsteine herabst"urzen.

Die mikroskopische Forschung entdeckte in diesen Steinen ein Gemenge kristallinisch k"orniger Metall- und Mineral-K"orper, vor Allem Eisen in Verbindung und Mengung mit Nickel, Kobalt, Titan, Kupfer, Zinn, Kiesel, Magnesium und anderen Stoffen. Manche Aërolithe bestehen fast g"anzlich aus metallischen Eisen und seinen Metalllegierungen, andere fast nur aus nicht metallischen Mineralk"orpern. Je nachdem die Eisenlegierungen die Hauptmasse bilden, mehr oder minder zusammenh"angend oder in K"ornern einem aus Quarz und Kieselverbindungen (sehr h"aufig aus Bronzit, Olivin und Augit) bestehenden Gemenge eingelagert oder letzteres mit Meteoreisenk"ornchen mehr oder minder gleichf"ormig gemengt erscheint, werden sie Pallasite oder Mesosiderite genannt. Eine dritte Klasse, die am h"aufigsten fallenden Meteorsteine, bestehen aus einer helleren oder dunkleren Grundmasse, welche gebildet wird aus einem Gemenge von K"ornchen von Meteoreisen, Magnetkies, Chromeisen, Titanit, Olivin, Augit, Bronzit, Anorthit, Quarz sc., in welcher Masse sich auf's Zahlreichste eingelagert finden kleinere oder gr"o"sere helle kugelige oder birnenf"ormige K"ugelchen, $\chi$o$\nu\delta\rho$o$\iota$ [chondroi], scheinbare Kristalldrusen von Kieselverbindungen, Bronzit oder Enstatit genannt. Diese mineralogisch schwierig zu charakterisierenden, in chemischer Beziehung sehr variablen Steine werden Chondrite genannt. Zuweilen sind diese Chondrite ganz schwarz und in ihnen wurden amorphe Kohle und bitumin"ose Stoffe als wahrscheinliche Zersetzungsprodukte organischer Verbindungen wahrgenommen, "uber deren Natur keine Vermutung gewonnen werden konnte.

"Uber diese Chondrite mit ihren mannigfachen undefinierbaren Einschl"ussen sind nun nicht nur Vermutungen, sondern Ergebnisse m"uhevollster Forschung, enthalten in einem Epoche machenden Werke: \emph{Die Meteorite (Chondrite) und ihre Organismen} von Dr. Otto Hahn"', welches k"urzlich die Laupp'sche Presse in T"ubingen verlie"s und die Ansicht "uber die Natur der Meteorsteine in ein neues ganz unerwartetes Licht stellt.

Viele meiner Leser werden sich der von dem selben Verfasser 1879 ver"offentlichten Mitteilung "uber \emph{Die Urzelle} d. h. "uber die einfachsten organisierten K"orper erinnern, welche derselbe in kristallinischen Gesteinen entdeckte. Wer hat dieses Buch gelesen und nicht, ungeachtet seiner zahlreichen Darstellungen der in jenen Urgesteinen gesehenen Pflanzen, gewisse Zweifel gehegt! Selbst in Meteorsteinen sollten Organismen, sollten pflanzliche Gebilde zu erkennen sein. Pflanzen, deren eine, den Algen und Farn verwandt, zu Ehren des deutschen Kaisers als Urania Guilielmi beschrieben und auf der 17. Tafel abgebildet wurde.

Ungeachtet mancher Widerspr"uche gegen diese seine Entdeckung, hat nun der seiner guten Sache bewusste Autor dieser beiden jetzt uns vorliegenden Abhandlungen sich nicht abhalten lassen, seine Entdeckung weiter zu verfolgen. Hunderte von D"unnschliffen mussten angefertigt, auf das Sorgf"altigste gepr"uft und mit einander verglichen werden, um das fr"uher gewonnene Resultat zu best"atigen und dahin zu erweitern, dass manche Meteorite --- und zwar nennt Hahn in der vorliegenden Schrift 18 verschiedene, von ihm untersuchte Meteorsteine aus der Reihe der Chondriten, deren Fallzeit genau bekannt ist --- fast g"anzlich aus einem Gemenge von Organismen bestehen. So ist es also auch hier das Mikroskop, welches, wie schon [Friedrich August von] Quenstedt (\emph{Handbuch der Mineralogie} S. 722) es vorhersagte, das R"atsel der Zusammensetzung der Meteorsteine l"osen musste.

Hahn gibt in seinen Beschreibungen der Organismen, welche er in den 18 von ihm untersuchten, aus den verschiedensten Gegenden der Erde stammenden Meteorsteinen auffand, solche aus der Klasse der Schw"amme, Nadelschw"amme, Korallen und Crinoiden, indem er zu dem Resultate kommt, dass die vermeintlichen Enstatite, Bronzite u. a. K"ugelchen dieser Chondrit-Meteorsteine nichts anderes sind als Organismen, deren Gewebe, gleich Korallen und Crinoiden, muschel- und schneckenschalig Mollusken sc. mit unorganischen Substanzen sich auf's H"ochste verband, so zu sagen mikroskopisch kleine Kiesel- und Kalk-Korallenst"ocke, Schw"amme sc., welche K"ugelchen nun die Hauptmasse des Gesteines bilden. Hahn meint gesunden zu haben, dass sowohl Individuen einer und derselben organischen Art dieser Chondrite aus verschiedenartiger Mineralsubstanz bestehen, bald der Zusammensetzung des Enstatites, bald der des Bronzites "ahnlich: als auch umgekehrt, dass eine und dieselbe Mineralsubstanz von den verschiedensten in dem Meteorsteine vorkommenden Organismen assimiliert worden sei und zum Ausbau ihres K"orpers gedient habe.

"Ubrigens d"urste auf die Eigenschaft, dass das Vegetations-Zentrum das scheinbare "`Kristallisation-Zentrum"' bei diesen K"orperchen stets exzentrisch liegt, als Unterscheidungsmerkmal von wirklichen Kristalldrusen nicht allzugro"ses Gewicht gelegt werden. Denn auch in Kristalldrusen liegt der Kristallisationsanfang h"aufig exzentrisch, und zwar ganz am Rande, wenn die Drusen sich sehr fr"uh auf einen festen K"orper niederlie"sen, etwas weniger exzentrisch, wenn dieses Festsetzen sp"ater geschah; ganz zentrisch liegt der Kristallisationsanfang der Drusen nur, wenn sich dieser schwimmend in einer Fl"ussigkeit bildete, wie es z. B. h"aufig bei organischen Substanzen vorkommt, weshalb auch die Oolithk"ugelchen als in einer Quelle, einem Sprudel, gebildet zu betrachten sind. Dass indessen die Entdeckung von Organismen in den seither f"ur Gl"aser (!!) oder Kristallisationen gehaltenen Chondriten richtig ist, bleibt f"ur den zweifellos, der mit den n"otigen Vorkenntnissen versehen sich mit der Untersuchung dieser Aërolithe besch"aftigte.

Eine vorz"ugliche, h"ochst genaue physikalische Beschreibung dieser Chondrite gibt [Karl Wilhelm von] G"umbel in seiner lehrreichen Abhandlung: "`"Uber die in Bayern gefundenen Steinmeteoriten"' (\emph{Sitzungsberichte der mathematisch physikalischen Klasse der k"onigl. bayrischen Akademie der Wissenschaften zu M"unchen} 1878), auf der einige S"atze hier zitiert werden m"ogen, um den Standpunkt zu kennzeichnen, den die Wissenschaft bis heute in dieser Frage einnahm.

"`"Uberblickt man die Resultate der Untersuchung dieser wenn auch beschr"ankten Gruppe von Steinmeteoriten, so dr"angt sich die Wahrnehmung in den Vordergrund, dass sie, trotz einiger Verschiedenheit in der Natur ihrer Gemengteile, doch von vollst"andig gleichen Strukturverh"altnissen beherrscht sind. Alle sind unzweifelhafte Tr"ummergesteine, zusammengesetzt auf kleinen und gr"o"seren Mineralsplitterchen, auf den bekannten rundlichen Chondren, welche meist vollst"andig erhalten, aber oft auch in St"ucke zersprungen vorkommen und aus Gr"aupchen von metallischen Substanzen, Meteoreisen, Schwefeleisen, Chromeisen bestehen. Alle diese Fragmente sind an einander geklebt, nicht durch eine Zwischensubstanz oder durch ein Bindemittel verkittet, wie sich "uberhaupt keine amorphen, glas- oder lavaartigen Beimengungen vorfinden. Nur die Schmelzrinde und die oft auf Kl"uften austretenden, der Schmelzrinde "ahnlich entstandenen schwarzen Ueberrindungen bestehen aus amorpher Glasmasse, die aber erst beim Niederfallen innerhalb unserer Atmosph"are nachtr"aglich entstanden ist. In dieser Schmelzrinde sind die schwerer schmelzbaren und gr"o"seren Mineralk"ornchen meist noch ungeschmolzen eingebettet. Die Mineralsplitter tragen durchaus keine Spur einer Abrundung oder Abrollung an sich, sie sind scharfkantig und spitzeckig. Was die Chondren anlangt, so ist ihre Oberfl"achenie gegl"attet, sie ist vielmehr stets h"ockerig uneben, maulbeerartig rauh und warzig oder facettenartig mit einem Ansatz von Kristallfl"achen versehen. Beile derselben sind l"anglich, mit einer deutlichen Verj"ungung oder Zuspitzung nach einer Richtung, wie es bei Hagelk"ornern vorkommt. Ost begegnet man St"uckchen, welche offenbar als Teile zertr"ummerter oder zersprungener Chondren gelten m"ussen. Als Ausnahme kommen zwillingsartig verbundene K"ugelchen vor, h"aufig solche, in welchen Meteoreisenst"uckchen ein- oder angewachsen sind. Nach zahlreichen D"unnschliffen sind sie verschiedenartig zusammen gesetzt. Am h"aufigsten findet sich eine exzentrisch strahlig faserige Struktur in der Art, dass von einer weit aus der Mitte nach dem sich verj"ungenden oder etwas zugespitzten Teile hin verr"uckten Punkte aus ein Strahlenb"uschel gegen Au"sen sich verbreitet. Da die in den verschiedensten Richtungen gef"uhrten Schnitte immer s"aulen- oder nadelf"ormige, nie bl"atter- oder lamellenartige Anordnung in der diesen B"uschel bildenden Substanz erkennen lassen, so scheinen es in der Tat s"aulenf"ormig Fasern zu sein, aus welchen sich solche Chondren aufbauen. Bei gewissen Schnitten gewahrt man, dieser Annahme entsprechend, in den senkrecht zur L"angenrichtung gehenden Querschnitten der Fasern nur unregelm"a"sig eckige, kleinste Feldchen, als ob das Ganze aus lauter kleinen polyedrischen K"ornchen zusammengesetzt sei. Zuweilen sieht es aus, als ob in einem K"ugelchen gleichsam mehrere nach verschiedener Richtung hin strahlende Systeme vorhanden w"aren, oder als ob gleichsam der Ausstrahlungspunkt sich w"ahrend ihrer Bildung ver"andert habe, wodurch bei Durchschnitten nach gewissen Richtungen eine scheinbar wirre, stengliche Struktur zum Vorschein kommt. Gegen die Au"senseite hin, gegen welche der Viereinigungspunkt des Strahlenb"uschels einseitig verschoben ist, zeigt sich die Faserstruktur meist undeutlich oder durch eine mehr k"ornige Aggregatbildung ersetzt. Bei keinem der zahlreichen angeschliffenen Chondren konnte ich beobachten, dass die B"uschel so unmittelbar bis zum Rande verlasen, als ob der Ausstrahlungspunkt gleichsam au"serhalb des K"ugelchens l"age, sofern nur dasselbe vollst"andig erhalten und nicht etwa blo"s ein zersprungenes St"uck vorhanden war. Die zierlich quergegliederten F"aserchen verlasen meist nicht nach der ganzen L"ange des B"uschels in gleicher Weise, sondern sie spitzen sich allm"alig zu, ver"asteln sich oder endigen, um andere an ihre Stelle treten zu lassen, so dass in dem Querschnitte eine mannigfache, maschenartige oder netzf"ormige Zeichnung entsteht. Diese F"aserchen bestehen, wie dies schon vielfach im Vorausgehenden geschildert wurde, aus einem meist helleren Kerne und einer dunkleren Umh"ullung, jener durch S"auren mehr oder weniger zerlegbar, letztere dagegen dieser Einwirkung widerstehend. H"ochst merkw"urdig sind die schalenf"ormigen "Uberrindungen, welche aus Meteoreisen zu bestehen scheinen und in der Regel nur "uber einen Kleineren Teil der K"ugelchen sich ausbreiten. Die gleichen einseitigen, im Durchschnitte mithin als bogenf"ormig gekr"ummte Streischen sichtbare "Uberrindungen kommen auch im Innern der Chondren vor. --- Doch nicht alle Chondren sind exzentrisch faserig; viele, namentlich die kleineren, besitzen eine feink"ornige Zusammensetzung, als best"anden sie aus einer zusammengeballten Staubmasse. Auch hierbei macht sich zuweilen die einseitige Ausbildung der K"ugelchen durch eine exzentrisch gr"o"sere Verdichtung der Staubteile bemerkbar. --- Der gew"ohnliche Typus der Meteorite von steiniger Beschaffenheit ist soweit "uberwiegend derjenige der sog. Chondrite, und die Zusammensetzung sowie die Struktur aller dieser Steine so sehr "ubereinstimmend, dass wir den gemeinsamen Ursprung und die uranf"angliche Zusammengeh"origkeit aller dieser Art Meteorite --- wenn nicht aller --- wohl nicht weiter in Zweifel ziehen k"onnen.

Der Umstand, dass die s"amtlich in h"ochst unregelm"a"sig geformten St"uckchen in unsere Atmosph"are gelangen --- abgesehen von dem zerspringen innerhalb der letzteren in mehrere Fragmente, was zwar h"aufig vorkommt, aber doch nicht in allen F"allen angenommen werden kann, namentlich nicht, wenn durch direkte Beobachtung das Fallen nur eines St"uckes konstatiert ist, --- l"asst weiter schlie"sen, dass sie bereits in regellos zertr"ummerten St"ucken als Abk"ommlinge von einem einzigen gr"o"seren Himmelsk"orper ihre Bahnen im Himmelsraume ziehen und in ihrer Zerstreutheit einzeln zuweilen in den Attraktionsbereich der Erde geratend zur Erde niederfallen. Der Mangel urspr"unglicher, lavaartiger, amorpher Bestandteile, in Verbindung mit der "au"seren unregelm"a"sigen Form, d"urste von geo- oder kosmologischen Standpunkte aus die Annahme ausschlie"sen, dass diese Meteorite Ausw"urfe von Mondvulkanen, wie vielfach behauptet wird, fein k"onnen. --- Es scheinen daher die Meteorite aus einer Art erstem Verschlackungsprozess der Himmelsk"orper, aber da sie metallisches Eisen enthalten --- bei Mangel von Sauerstoff und Wasser hervorgegangen zu sein."'

Unser Autor schlie"st sich diesem Urteil "uber die Aggregatform der Meteorite vollkommen an, jedoch mit dem Vorbehalte, dass, wie gesagt, jene kleinen kugelig-birnenf"ormigen K"orperchen, welche der haupts"achlichste Gemengteil der Steinmeteorolithe ausmachen, nicht Mineralindividuen sind, sondern nur Organisiertes, ebenso wie auch fast die gesamte mit Rissen und Spr"ungen durchsetzte quarzige Grundmasse. Im Gegensatze zu den von G"umbel beschriebenen Meteoriten findet sich in dem von Knyahinya eine geringe quarzige, zersprungene Zwischensubstanz. "`Alles Leben"' ein Urwald oder vielmehr ein Polypen-, ein Spongienwald im Kleinen, ein Chaos von aufeinander gewachsenen Formen, den heutigen zum Verwechseln "ahnlich, nur Alles unendlich kleiner.

Auf 32 photographischen Tafeln werden in 142 Abbildungen eine Unzahl der entdeckten Organismen abgebildet, neben anderen der irdischen Sch"opfung, die zum Vergleiche herangezogen wurden. Leider hat sich unser Autor durch einen kritiklosen Kritiker verleiten lassen, seine in der \emph{Urzelle} befolgte Methode des Selbstzeichnens aufzugeben und statt eigener Zeichnungen nur Photographien zur Erl"auterung und Beglaubigung vorzulegen; Beides nebeneinander w"urde den Leser mehr befriedigt haben! Denn so naturgetreu auch photographische Bilder einen bestimmten Zustand, eine bestimmte Fl"ache, die gerade im Fokus des Mikroskops sich befindet, wiedergeben, falls Licht und Farbenverh"altnisse g"unstig sind, so wenig gen"ugen sie einem Beobachter dem Zwecke, ein Bild dessen zu geben, was er in einem bestimmten Zustande des untersuchten Objektes f"ur das Charakteristische h"alt, wozu h"aufig perspektivische Zeichnungen dessen, was derselbe durch Ver"anderung des Fokus (der Sehweite) erkennen konnte, nicht zu entbehren sind.

Die umstehende Zeichnung eines l"angsdurchschnittenen, drusenartigen K"ugelchens ist von mir mit Hilfe eines in der Darstellung naturhistorischer, insbesondere mikroskopischer Gegenst"ande erfahrenen und ge"ubten K"unstlers, des Herrn Professor [Friedrich Eduard] Metzger hierselbst, mit gr"o"ster Sorgsamkeit angefertigt worden. Nach reiflichster "Uberlegung haben wir das, dem Objekte wirklich Eigent"umliche von dem Zuf"alligen, d. h. von den "au"serlich adh"arierenden, von dem durch Lichtbrechung verursachten Scheine zu sondern gesucht; es kam hierbei zun"achst daraus an, nachzuweisen, dass das Objekt organisiert sei. Ich glaube, es wird uns besser und vollst"andiger gelungen sein, als dem Photographen, so vollendet auch dessen Bilder, dem Zustande der photographischen Technik entsprechend, von verschiedenen Exemplaren dieses Organismus auf Taf. 1, 8, 9, 10, 11, wiedergegeben sind. Das, ungeachtet der Zartheit des Schliffes, z. T. oberseits hie und dort das Objekt bedeckende Fremdartige, ferner Risse, die ich f"ur zuf"allig, durch die Operation des S"agens und Schleifens hervorgebracht hielt, haben wir nicht mitgezeichnet, um das komplizierte stark vergr"o"serte, dennoch minuti"ose Bild nicht mit Nebens"achlichem zu beladen. Vielleicht sind Strukturverh"altnisse, die zu dem obliegenden Beweise h"atten dienen k"onnen, dass das Objekt ein organisierter K"orper ist, aus zu gro"ser Vorsicht fortgelassen worden, z. B. hier und dort eine querlaufende schr"agstehende Scheidewand in der einen "astigen Faser; wir hielten sie f"ur gleichwertig mit anderen gleichlaufenden Linien, die uns zuf"allige Risse zu sein schienen. Mit einem Worte: das Bild gibt das, was ich dem Leser als von mir an dem Organismus beobachtet zeigen will, es soll eine lange schwer verst"andliche Beschreibung ersetzen.

Dieser dargestellte K"orper stammt aus einem bei Knyahinya in Ungarn am 9. Juni 1866 gefallenen Steinmeteore, welches z. T., das hei"st in einem 27 Pfd. schweren St"ucke von dem Beobachter des Falles noch lauwarm ausgenommen, demselben einen 3 Tage anhaltenden, durchdringenden Knoblauchgeruch (Selen ?) mitteilte. Der Stein kam mit Donnergenkrach aus einer Wolke als gl"uhende Kugel mit langem Schweife, aus welchem nach allen Seiten kleinere herausfuhren. Ein gro"ser 5$^{1/2}$ Zentner schwerer Block war gleichzeitig 11$^{\prime}$ tief in den Boden einer Wiese eingedrungen.

Dieser Organismus ist von Hahn als Koralle bestimmt worden; er ist sehr "ahnlich dem in den "altesten silurischen Schichten unserer Erdrinde vorkommenden \emph{Favosites}, wie [Georg August] Goldfu"s auf seinen Tafeln 26 und 27 diese Korallen abbildet; ebenso der von [Georg Amadeus Carl Friedrich] Naumann auf der ersten Tafel seines Handbuches gezeichneten silurischen \emph{Calamopora}. Ich w"ahlte diesen K"orper zur Darstellung unter den zahllosen Bruchst"ucken von Geweben, --- die in ihrer gro"szelligen Struktur sich leicht als Pflanzengewebe zu erkennen gegeben haben w"urden, --- weil er einen der Chondritenk"ugelchen bildet, denen die Mineralogen ihre besondere Aufmerksamkeit schenkten; K"ugelchen, welche die chemische Analyse als eine Art Bronzit (Enstatit) nachwies, und die wegen ihrer Kristalldr"usenform und ihrer stenglichen Struktur mehr wie alle "ubrigen einem kristallinischen K"orper gleichen. Das gezeichnete Individuum ist ein fast mittlerer L"angenabschnitt einer jener birnenf"ormigen K"orperchen; die oberen und unteren Teile sind weggeschlissen, die R"ander z. T. von dem als Grundmasse dienenden Eisensilikate durchdrungen, im "Ubrigen ist der ganze Organismus durch und durch verkieselte oder in jene Enstatit genannte Kieselverbindung ver"andert. Er besteht aus fast geraden, schwach radialen, nach dem peripherischen Ende etwas erweiterten R"ohren, die selten, wie in Figur 2, eine Verzweigung erkennen lassen, wie es scheint, ohne Scheidew"ande, wenigstens in den j"ungeren Teilen; in den unteren engeren Enden vielleicht mit rechtwinkelig auf der L"angenwand stehenden Scheidew"anden. Einzelne Partien dieses R"ohrensystemes sind, mitten zwischen den fast parallelen, etwas gebogen und scheinen in eine verd"unnte abgerundete Spitze zu enden. Alle die R"ohren sind, wie es mir schien, und wie in dem unter Fig. 1 bei b gezeichneten St"uckchen dargestellt, angef"ullt mit einer Reihe kugeliger, dickwandiger Zellen, die in den "alteren Teilen unmittelbar aneinander liegen, w"ahrend in den j"ungeren Enden die R"ohrenhaut verh"altnism"a"sig dicker zu sein scheint, wohl aber noch eine L"angenh"ohlung, ein Lumen, zu erkennen ist, in welchem in regelm"a"sigen Abst"anden kleine, dunkelumrandete Bl"aschen liegen, wie in a unter Figur 1 gezeichnet. Die "Ubergangsformen zwischen diesen beiden Inhaltsanteilen der R"ohren konnte ich nicht genau erkennen. Zwischen den R"ohren befindet sich eine tr"ube dunkelgelbe bis braune Masse, in der aber gleichfalls eine Reihe von hellen Bl"aschen zu erkennen ist; vielleicht die Inhaltsbl"aschen dar"uber liegender, zum gr"o"sten Teile weggeschlissener R"ohren. Wie gesagt, bestimmte Hahn diesen K"orper als \emph{Favosites}, indem er diese scheinbaren Bl"aschen f"ur durchschnittene Kan"ale, sog. Knospenkan"ale h"alt. In der Tat hat derselbe, abgesehen von der au"serordentlichen Kleinheit, die gr"o"ste Ähnlichkeit mit den Abbildungen oben genannter Korallen; ich halte dieselben, nach dem einen, mir zur Ansicht vorliegenden Exemplare, f"ur eine farblose Fadenalge, f"ur eine Hysterophyme, z. B. f"ur \emph{Leptomitus} oder \emph{Leptothrix}; ohne hinreichendes Material, wie es heute nur Hahn selbst zu Gebote steht, und wie es derselbe auf das Flei"sigste benutzte, w"are es aber ein zu gewagtes Unternehmen, eine von der seinigen abweichende Meinung aufstellen zu wollen.

Jedenfalls ist dieser K"orper keine Druse nadelf"ormiger oder s"aulenf"ormiger Kristalle, wie bisher die Mineralogen meinten, sondern ein organisiertes Gebilde; denn wirkliche Kristalle, die aus verdunstenden oder abk"uhlenden L"osungen sich ausscheiden, sind strukturlos und homogen.

Von gr"o"stem Interesse f"ur die Aufkl"arung der Natur dieser Organismen der Meteoriten sind h"ochst "ahnliche von Paul. F. Reinsch k"urzlich in der Steinkohle entdeckte Gebilde; eine Entdeckung, die zu meiner Kenntnis zu bringen der Herr Herausgeber die G"ute hatte.

Nach Reinsch's Beobachtung bestehen einzelne Schichten der s"achsischen Kohle zu 20\% aus solchen Organismen, ebenso wie die Chondriten-Meteorite zum gr"o"stem Teile aus ihnen zusammengesetzt sind. Auch die von Reinsch entdeckten Pflanzen sind h"ochst kleine, mikroskopische Gebilde, auch sie kommen in wenigen Formen, aber in gr"o"ster Anzahl beisammen als Grundlage der betreffenden Kohlenfl"otze vor; z. T. bestehen sie, gleich dem in Figur 1 und Figur 2 gezeichneten Organismus, aus ver"astelten, mehr oder minder freie Zellen enthaltenden, konzentrischen Fasern. Reinsch h"alt sie f"ur Algen und Pilze, etwa f"ur Schleimpilze, indem auch er, gest"utzt auf vollg"ultige Gr"unde, ausdr"ucklich gegen unorganische Natur derselben protestiert. Auch darin stimmen diese Kohlenorganismen mit denen der Meteoriten, dass sie in ihren "alteren Teilen vererzen (in Schwefelkies) oder verkieseln. Ich halte auch diese Organisationen der Steinkohle f"ur Hysterophymen der die Kohle zusammensetzenden absterbenden, verwesenden Pflanzen: f"ur Hysterophymen, deren Natur und Entwickelung ich wiederholt --- zuletzt in meiner \emph{Deutschen medizinischen Flora} 1880 --- beleuchtete; Organisationen, die jeder vorurteilsfreie, sorgf"altige Beobachter auf die von mir angegebene Weise aus pflanzlichen und tierischen Gewebezellen, als Metamorphosen derselben sich entwickeln sehen kann. In dem von Reinsch entdeckten Falle geschieht die nekrobiotische Metamorphose unter Wasser, in jenem von Hahn entdeckten in feuchtigkeitsschwangerer Atmosph"are; in beiden F"allen sind es die einfachen Zellenvermehrungsformen, wie sie und das Studium der Kontagien und Miasmen kennen lehrte und wie ich sie in meiner \emph{F"aulnis und Ansteckung 1872} darstellte.

Hahn fand nun ferner, dass alle von ihm untersuchten Steinmeteorite, und nur "uber solche "au"sert er sich in dem vorliegenden Werke, die gleichen organisierten Gesch"opfe enthalten. Ein Resultat, welches schon die mineralogische Untersuchung in Bezug auf chemisch-physikalische Verh"altnisse derselben erlangt hatte; und diese Tatsache f"uhrt ihn S. 44 zu dem Schul"se: "`alle diese Chondrite als Tr"ummer, welche nach der Zerst"orung des Planeten kreisten, bis sie gl"ucklicherweise in den Fallkreis unserer Erde kamen."'

Die Formen der bis jetzt in den Chondriten erkannten Gesch"opfe geh"oren alle dem Wasser an; die ganze Masse dieser Meteorite scheint unter Wasser gewachsen zu sein, die zahllosen, mikroskopischen Organismen versteinerten entweder nachtr"aglich oder, was die chemische Analyse der verschiedenen K"orperchen wahrscheinlicher macht, jedes einzelne verband sich in seiner Weise mit den im Wasser gel"osten Mineralsubstanzen, assimilierte dieselben, gleich den jetzt noch lebenden Muscheln, Korallen, Bazillarien, Equiseten, verschiedenen Verbenazeen sc., deren H"aute gleich den Knochen der Wirbeltiere verkieseln und verkalken. Schlie"slich wurden sie dann in dem R"uckstande der eingetrockneten Kieselreichen N"ahrstofffl"ussigkeit mit einander verkittet und als quarzige Masse in ein zusammenh"angendes Gestein umgewandelt. Man sieht daher auch so zahllose kleinste durchscheinende und durchsichtige Organisationen --- in diesem Meteorsteine von Knyahinya wenigstens --- "ubereinander geh"auft, dass dies die Erkennung der eigentlichen Form der meisten von ihnen sehr erschwert, dass selbst ihr Vorhandensein f"ur diejenigen, welche mit den mikroskopisch-organischen Formen nicht vertraut sind, schwierig wahrzunehmen ist.

Den einzelnen organisierten K"ugelchen und Gewebebruchst"ucken zwischengelagert, findet sich, wie gesagt, eine, wenn auch geringe, Kieselmasse und in dieser zerstreut sind gr"o"sere und kleinere Splitter metallischen Eisens und Nickel-, Titan- oder Chrom-Eisen-Verbindungen, die z. T., wie es scheint, in das Silikat "ubergehen und auch die Organismen hie und da teilweise durchtr"anken, z. T. aber, d. h. die metallischen Eisenlegierungen, in scharfkantiger und unregelm"a"sig eckiger Form vorliegen. Die Entstehungsweise dieser metallischen Eisensplitterchen kann, wenn sie auf die vegetative T"atigkeit der Organismen bezogen wird, wie Hahn dies f"ur naturgem"a"s h"alt, indem er sich dabei auf die von mir in dieser Richtung gemachten Versuche und Beobachtungen st"utzt, eine doppelte sein: entweder kann das Metall durch die Sekrete derselben aus seinen L"osungen von Kiesel- oder Chlor- oder Chrom- sc. Eisen reduziert und metallisch auf dieselben niedergeschlagen sein, wie dies aus Silber- und Quecksilbersalzen durch Pilzvegetationen geschieht; oder es ist, gleich den Erden und Alkalien, gleich Natron, Kali, Kalk, Magnesia sc. von der assimilierenden Zellmembran aufgenommen und zur eigentlichen Konstitution derselben verwendet worden, indem diese Membran nach und nach immer h"oher basische Verbindungen formte und endlich ihre urspr"unglichen organischen Elemente g"anzlich ausgeschieden wurden\footnote{\frakfamily{Eine ausf"uhrliche Darstellung der assimilierenden, organisierenden T"atigkeit der lebendigen Zellmembran gab ich vor Kurzem (1880) in meiner \emph{Botanik} S. 17 - 22.}}, so dass, gleich reinen Magnesia- oder Kalksalzen, reine Metalllegierungen "ubrig blieben. Die Organismen der Letztwelt geben uns bis jetzt nur die ersten Entwicklungsstufen dieser Metallverbindungen als Anhaltspunkte f"ur diese Theorie, wie ich deren in meiner auch von Hahn ber"ucksichtigten Abhandlung "uber "`Chemismus der Pflanzenzelle"' niederlegte. Diese Organismen der Meteoriten lassen jedoch, durch die au"serordentliche Kleinheit, in der sie meistens auftreten, auf m"oglicherweise andere, von den heutigen verschiedene physikalische Verh"altnisse ihrer Entstehung schlie"sen, vielleicht auf eine bedeutend h"ohere oder niedrigere Temperatur sc. Wie sich unter solchen uns unbekannten Verh"altnissen nun die assimilierende Zellenmembran die unorganischen Elemente aneignet, das ist uns bis jetzt noch g"anzlich unbekannt. Dass die Organismen bei h"oherer Temperatur, z. B. bei der des Siedepunktes des Wassers, in viel kleinerer Form weiter vegetieren und sich vermehren, das erw"ahnte ich schon in der genannten Abhandlung "uber "`Chemismus der Pflanzenzelle"'. Inzwischen "uberzeugte ich mich nun, dass selbst bei noch h"oherer Temperatur, d. h. bei 150$^{\circ}$, die Lebenskraft der pflanzlichen Organisation nicht v"ollig erlischt, dass vielmehr die Inhaltszellchen einzelner Gewebezellen auch dann noch, wenn gleich sp"arlich sich entwickeln k"onnen, aber in ungew"ohnlich zarter und kleiner Form. Anderseits vermehren sich Organismen auch noch bei niedrigen, unter dem Gefrierpunkte liegenden Temperaturen, und auch dann in bedeutend geringerer Gr"o"se, als bei +30 bis 35$^{\circ}$ C. Dass sich Bakterien eine Stunde lang bei einer Temperatur von -100$^{\circ}$ C. lebend erhielten, wurde wiederholt beobachtet; k"onnte der Versuch lange genug fortgesetzt werden, so w"urde man vielleicht auch dann jenes Gesetz der Formverkleinerung best"atigt sehen.

Jedenfalls fordert das vorliegende Buch von Hahn durch die gl"anzende Entdeckung einer in den Meteoriten zur Erde gebrachten neuen Welt von Organismen zur Revision vieler, uns schon als sichere Ergebnisse der Beobachtung und Berechnung erschienener Lehrs"atze auf. Erkennen wir die so annehmbar erscheinende Vermutung, die Meteorite seien Teile von Kometen, als richtig an: so k"onnen die Kometen nicht gl"uhend fl"ussige, am Umkreise nur erkaltete und in einzelne Bruchst"ucke getrennte K"orper sein; denn diese Steinmeteore sind, vor dem Zusammentresen mit unserer Atmosph"are, nicht auf bedeutende W"armegrade erhitzt gewesen, sie w"urden zu Glas geschmolzen sein! Man erkennt aber nur eine geringe Einwirkung der W"arme --- vielleicht, wie fr"uher angedeutet, der Reibungsw"arme gegen die atmosph"arische Luft w"ahrend ihres Falles --- auf der "au"seren Oberfl"ache als wenige Linien dicke Rinde um jeden der herabgefallenen Steine. Diese Schmelzrinde bildet sich, wie es scheint, gr"o"stenteils erst nach dem h"aufig beobachteten und vernommenen Zerplatzen der die Leuchtkugel formenden ganzen Masse: denn jedes einzelne so entstandene kantige St"uck ist ringsum mit einer, wie es scheint gleich dicken Schmelzrinde umh"ullt; diese entstand demnach erst in den unteren dichteren Regionen der Atmosph"are. Geh"orten nun dennoch diese Meteorsteine urspr"unglich einem Kometen an, so befindet sich dieser nicht in geschmolzenem feurig-fl"ussigen Zustande, sein Licht ist ein erborgtes, ein reflektiertes und seine Masse von solcher Beschaffenheit, dass sie durch die empfangene W"arme nicht zum Schmelzen oder auf eine H"ohe erhitzt wurde, welche dass Leben von Organismen unm"oglich machen w"urde. Der Idee Hahn's und der Neptunisten "uber die Entstehung unserer Erde w"urde es entsprechen, sich den Kern der Kometen nicht feurig-fl"ussig, sondern w"assrig-fl"ussig, und seine in St"uckchen zersplitterte Rinde als Verdunstungs-Rinde zu denken. Denn wahrscheinlich war "`der erste Anfang unseres und daher aller Planeten eine organische Bildung (S. 40), --- die Zell; sie erh"alt ihn, so lange noch ein Lichtstrahl die Erde trifft!"' S. 50.

Aber auch an die oben schon ber"uhrte Idee des terrestrischen Ursprunges der Meteorsteine m"ochte wieder erinnert werden, an die historisch beglaubigten von Feuerkugeln und Meteorolithen begleiteten Staubregen; m"ussten nicht auch in diesem Falle diese Meteorolithen zusammengeschmolzene Gl"aser sein, wenn diese K"orper etwa erst in der Atmosph"are aus Passatstaub entstanden?

Nach der Anschauung Hahn's ist die ganze feste Masse der uns bekannten Himmelsk"orper das Produkt organisierender T"atigkeit; nach Hahn formen sich aus dem Chaos der Elemente zun"achst Zellen, die sich neben sog. organischen Elementen (Kohlenstoff, Sauerstoff, Wasserstoff, Stickstoff) auch in gr"o"ster Menge unorganische Elemente, d. h. Erden und Metalle aneignen und in ihre eigene Masse aufnehmen. Dieser energische, durch die ganze dampff"ormige und fl"ussige Masse der sich formenden Himmelsk"orper verbreitete Vegetationsprozess der Organismen k"onnte auch das von ihnen uns gesandte Licht hervorbringen, "ahnlich, wie wir es von einigen leuchtenden Tieren, Pflanzen und Hysterophymen (Spaltpilzen) unserer Erde kennen, das demnach dort st"arker ergl"anzen w"urde, wo sich die dasselbe erzeugenden Organismen in gr"o"serer Menge beisammen finden.

Dass diese mit organisierten K"orpern durchsetzten Meteoriten vor dem Zusammentreffen mit unserer Atmosph"are keine Schmelztemperatur zu ertragen hatten, zeigt auch ihre mittelst des Mikroskops erkannte Struktur zweifellos. Demnach kamen sie in ungeschmolzenem kalten Zustande in unsere Atmosph"are; an einem anderen, uns unbekannten Orte in der Ferne gebildet, wie sie uns jetzt vorliegen.

Vielleicht ist auch die Idee des kosmischen Ursprunges, wenigstens f"ur diese Art von Meteoriten, zu verlassen und wieder auf deren Entstehung als Konglomerate von Meteorstaub oder Passatstaub "ahnlicher Materie zur"uckzugehen, wie sie schon von [Pieter van] Musschenbroek, Dominikus Tata, [Eugène Louis Melchior] Patrin, [Ernst Friedrich] Wrede, Egen, v. Hof, Kesselmeyer u. A. verteidigt wurde, obgleich das Entstehen eines solchen Konglomerates mit unseren heutigen physikalischen Kenntnissen und Erfahrungen noch nicht bis ins Einzelne verfolgt werden kann.

Diese eben genannten Autoren, vorz"uglich Kesselmeyer, betrachten die Leuchtkugeln und die aus diesen herabfallenden Meteorite als atmosph"arische Sublimationsgebilde der von unseren Vulkanen ausgehauchten Mineral d"ampfe; und allerdings hat sich dem analysierenden Chemiker die Fl"uchtigkeit aller Mineralstoffe zum gro"sen Nachteile der quantitativen Analyse, bevor diese Eigenschaft fester K"orper hinreichend erkannt worden war, nur zu h"aufig in bedauerlicher Weise bemerkbar gemacht.

"Uberdies kennt jeder Besucher t"atiger Vulkane die interessante Erscheinung des kontinuierlichen Dampfes dieser zur Nachtzeit oft leuchtenden Feuerberge. Mit dem Wasser zugleich, welches den gr"o"sten Teil dieses Dampfes bildet, entquellen dem Krater auch best"andig feste, pulverf"ormige oder dampff"ormige Bestandteile des Gesteines, welches von den gl"uhenden Wasserd"ampfen durchzogen wird: pulverf"ormige Massen, sog. vulkanische Asche, denen sich zur Zeit der h"ochsten T"atigkeit mehr oder minder umfangreiche Gesteinsbruchst"ucke und geschmolzene Gesteine beimischen. Letztere fallen mehr oder minder bald zur Erde zur"uck, aber die staubf"ormigen Anteile werden mit dem Wasserd"ampfe bis zu erstaunlicher H"ohe mitgerissen, um sich in den oberen Regionen der Atmosph"are zu verteilen. Mit gro"sem Genuss betrachtete ich das anziehende Schauspiel, welches mir in den Kordilleren der Puracé gew"ahrte durch die gegen 5000$^{\prime}$ hohe Dampfs"aule, welche in der ruhigen Atmosph"are senkrecht in die H"ohe quoll, anfangs st"urmisch aus dem Gipfelkrater hervorwirbelnd, dann nach und nach langsamer steigend, bis sie, in gewisser H"ohe angelangt, sich waagrecht ausbreitete und eine Wolkenschicht bildete, die sich in den oberen Luftschichten an den R"andern wieder ausl"oste. Wie anderseits Staubmassen von der Oberfl"ache des Erdbodens senkrecht in die H"ohe wirbeln, auch gr"o"sere leichte K"orper, trockene Bl"atter, Schmetterlingsfl"ugel sc. mit sich f"uhren bis zu H"ohen, wo sie dem Auge entschwinden, sah ich besonders in den hei"sen Tiefebenen der Aequatorialgegend zur Zeit der Jahreswende, wenn sich hie und dort leichte W"olkchen bilden, deren wenig umfangreiche, auf den erhitzten, trockenen Boden der abgebrannten Llanos geworfene Schatten, eine stellenweise geringe Abk"uhlung desselben bewirken, hinreichend, die Entstehung der aufstrebenden Luftwirbel zu veranlassen, welche mit dem W"olkchen vorw"artsschreitend die Ebenen abfegen und die leichten Staubteile derselben himmelw"arts f"uhren, bis sie dem Auge entschwinden. Wie gro"se Massen auf diese Weise in den oberen Regionen der Atmosph"are angesammelt werden, um in oft sehr entfernten Gegenden sich wieder zu senken, das lehren die oben ber"uhrten Erscheinungen des Meteor- und Passatstaubes, die das Mikroskop als Mischung organisierter und unorganisierter K"orperchen nachwies. Dass die organisierten noch lebensf"ahigen Anteile dieses Staubes, wenn derselbe sich in der Atmosph"are mit feuchten Luftschichten mischt, wieder erwachen und ihre Lebens"au"serungen, ihre Assimilationst"atigkeit fortsetzen k"onnen und werden, wie wir ja die Entstehung der Bakterien und ihre Verwandten kennen und wie sie in der feuchten Kammer des Mikroskopikers beobachtet werden kann, ist wohl nicht zu bezweifeln; bis wie weit aber die Gestaltungsprozesse dieser, in den lustigen kalten H"ohen weitergetragenen, mikroskopischen Zellchen fortgef"uhrt werden k"onnen, dar"uber haben wir bisher noch keine Ahnung, w"urden eine solche vielleicht aus den "uberraschenden Mitteilungen Hahn's sch"opfen k"onnen, w"are uns nicht der Kondensations-Akt solcher mit Abk"ommlingen des Passatstaubes geschw"angerten Wolken noch durchaus r"atselhaft und wir deshalb im Zweifel, ob wir diese Erscheinungen in Zusammenhang bringen d"urfen.

Dass ungeheure Massen, die sicher der Erdatmosph"are entstammen, sich in deren Bereich koagulieren k"onnen, beweisen die Eismassen, die zuweilen auf die Erde herabfallen. Ich selbst beobachtete einmal einen Hagelschauer in S"udbaiern, dessen K"orner die Gr"o"se von H"uhnereiern hatten, und diese waren nicht abgerundet, wie gew"ohnliche Hagelk"orner, sondern scharfkantige St"ucke, wie es schien, Bruchst"ucke gr"o"serer Massen; eine Erscheinung, die auch Delcro"s beobachtete. Diese scharfkantigen Eisst"ucke erinnern dringend an das Bersten der Steinmeteorite in der Erdn"ahe. Im Jahre 1802 am 28. Mai fiel bei Puztemischel in Ungarn w"ahrend eines Hagelwetters ein Eisklumpen zur Erde, der 3$^{\prime}$ L"ange, 3$^{\prime}$ Breite und 2$^{\prime}$ Dichte hatte; sein Gewicht wurde auf 11 Ztr. gesch"atzt. L. v. Buch berichtet aus Heyne's \emph{Tracts historical and statistical on India} von einer Eismasse, die bei Seringapatam in Indien fiel und die Gr"o"se eines Elefanten hatte, so dass sie, ungeachtet der gro"sen Hitze dieses Landes, eines Zeitraumes von 2 Tagen bedurfte, um geschmolzen zu werden. Diese Eismassen entstehen durch Gefrieren von Regenwolken in Folge der pl"otzlich erk"altenden Einwirkung heftiger trockener Luftstr"ome. In solchen Hagelkornern wurden selbst Metallkerne beobachtet; so bei Mayo in Irland am 21. Juni 1821. K"onnten vielleicht auch durch Aufeinandertreffen von verschiedenen mit Mineralgasen und Organismen geschw"angerten Luftstr"omungen in den h"ochsten Regionen der Atmosph"are jene Chondritmassen sich koagulieren? Die am 14. Juli 1860 bei Dhurmsala in der Gegend von Lahore unter Explosion herabgefallenen Steine sollen, obgleich sie an der Oberfl"ache geschmolzen waren, dennoch so kalt gewesen sein, dass Personen, welche sie ausheben wollten, sie nicht in der Hand behalten konnten, weil sie vor K"alte ein Kriebeln in den Fingern bekamen. Brachten nun diese Steine die K"alte des Weltraumes oder die Temperatur der oberen Erdatmosph"are zu den Menschen herab? Diese Wahrnehmung an den Meteoriten bei Dhurmsala erhielt k"urzlich ein Seitenst"uck in dem von Thomas Carnalley im Vakuum bis auf +180$^{\circ}$ C. erhitzten Eiszylinder.

Schon manche Erscheinungen beim Fallen der Meteorsteine machen ihre Natur als kosmische K"orper zweifelhaft und erinnern an die dichten Wolken von Passatstaub, die sich hin und wieder in Europa und Asien als Massen von Millionen von Zentnern niederlassen und der Westk"uste Afrika‘s die Benennung "`Nebelk"uste"' dem benachbarten Ozeane die des Meeres der Finsternisse eintrugen. Ehrenberg fand dergleichen Passatstaub aus Minertr"ummern, organisierten Fragmenten von Land- und S"u"swasserformen zusammengesetzt. Sollte, trotz aller Zweifel der Physiker, dennoch ein Teil der Meteoriten solchen Staubwolken ihre Entstehung verdanken? Dies wiederum ungepr"uft von der Hand zu weisen, w"urde uns fast in denselben Fehler verfallen lassen, den die Mitglieder der Pariser Akademie der Wissenschaften Jahrzehnte hindurch sich zu Schulden kommen lie"sen, wie sie diejenigen als Toren abwiesen, die, als Augenzeugen, ihnen aus den Wolken oder vom Himmel gefallene Steine "uberbrachten.

Verschiedene Tatsachen und Beobachtungen sprechen daf"ur, dass die Leuchtkugeln erst innerhalb der Atmosph"are aus dampff"ormigen K"orpern sich verdichten, dass aus Wolkenmassen sich feste K"orper bilden k"onnen. So sahen Landleute am 14. Juli 1847 Morgens 3$^{3/4}$ Uhr in der Gegend von Braunau am nordwestlichen Horizonte ein schwarzes W"olkchen aufsteigen, welches pl"otzlich ergl"uhte, nach allen Seiten Blitze und unter Donnerget"ose zwei feurige Streifen zur Erde sandte. An dem unteren Ende des einen dieser Streifen fand ein Bauer ein frisch in die Erde gebohrtes Loch, aus welchem nach sechsst"undiger Arbeit ein 47 Pfund schweres Eisenst"uck hervorgezogen wurde, welches noch so gl"uhend war, dass man sich noch die Hand daran verbrennen konnte. Der zweite Streifen war in ein Haus eingeschlagen, in welchem eine, jenem "ahnlich, 34 Pfd. schwere Eisenmasse gro"se Verw"ustungen angerichtet hatte. Auch der oben eingehend besprochene Meteorit von Knyahinya soll aus einer furchtbar krachenden Wolke gefallen sein, die in der Ferne als gl"uhende Kugel mit Schweif erschien, aus welcher nach allen Seiten kleine Kugeln hervorspr"uhten. Dergleichen Beobachtungen werden noch mehrere angef"uhrt, und scheint es etwas voreilig, sie alle auf ein zuf"alliges Zusammentresen von herabfallenden Sternschnuppen oder Leuchtkugeln mit heraufziehenden W"olkchen erkl"aren zu wollen.

Die Reibung solcher pulvrigen Masse, wie sie als Passatstaub beisammen vorkommen, erzeugt ohne Zweifel elektrische Spannung und k"onnte diese wohl eine Vereinigung desselben veranlassen, eine Vereinigung, die bei Gegenwart gen"ugender Mengen von Wasserdampf vielleicht ohne eigentliche Schmelzung vor sich geht.

Dass die vorausgesetzte Reibung, in die Erdatmosph"are gelangter K"orper gegen diese Atmosph"are allein nicht gen"uge, das Leuchten und die Erhitzung der Meteoriten zu erkl"aren, darauf hat schon 1835 v. Hof aufmerksam gemacht, indem dasselbe nicht in den obersten, d"unnsten Luftschichten beginnen und in den untersten dichtesten erl"oschen w"urde, vielmehr bis zum Erreichen der Erdoberfl"ache mit der stets zunehmenden Fallgeschwindigkeit best"andig zunehmen m"usse.

Auf die Verschiedenartigkeit der Sternschnuppen und Leuchtkugeln deutet schon die au"serordentliche Verschiedenartigkeit der Fallgeschwindigkeit beider Meteore. W"ahrend die Sternschnuppen mit einer Geschwindigkeit von 10--20 Meilen in der Sekunde das Firmament durcheilen, bewegen sich die viel gr"o"seren Leuchtkugeln nur mit einer Geschwindigkeit von 1 oder wenige Meilen in der Sekunde. Die aus denselben herabgefallenen Eisen-Meteorite kommen zuweilen noch im halbfl"ussigen, geschmolzenen Zustande gl"uhend hei"s zur Erdoberfl"ache, so dass sich Kieselsteine in dieselben hineindr"ucken, was z. B. 1808 bei Parma [Borgo San Donino] und bei Belaja Zerkara [Bjelaja Zerkov] in Russland beobachtet wurde. Auch die Steinmeteore hat man im halbweichen Zustande zur Erde gefallen angetroffen, so z. B. bei Cold Bokkeveld auf dem Kap der guten Hoffnung, wo am 13. Okt. 1838 aus einer Feuerkugel, unter heftigen Explosionen viele, anfangs weiche, schwarze, kohlige, beim Anhauchen ammoniakalisch riechende, vom Wasser und bitumin"oser Substanz durchdrungene Steine von zusammen mehreren 100 Pfd. Gewicht noch weich zur Erde gelangten und erst sp"ater erh"arteten. Ähnlich verhielt sich ein Stein, der 1864 bei Orgueil zur Erde fiel; er war weich und zwischen den Fingern zerdr"uckbar; nur die Schmelzrinde und ein Zement l"oslicher Salze hielt ihn zusammen. Sollten Erscheinungen so verschiedener Natur: Leuchtkugel, die einmal halbfl"ussige feurige Metallmassen, ein andermal w"asserig-weiche Erdkonglomerate zur Erde senden, nicht vielleicht v"ollig verschiedenen Vorg"angen ihre Entstehung verdanken? Leuchtkugeln und Sternschnuppen eine verschiedene Abstammung haben?

Vieles bleibt hier noch zu beobachten; zun"achst, nach Hahn's Vorgange, alle Meteorsteine nochmals gr"undlich zu untersuchen.

W"are auch nur dies das Resultat der Hahn'schen Arbeit, so w"urde demselben der Dank der Wissenschaft f"ur diese Anregung geb"uhren; so aber ist sein Verdienst durch die Entdeckung der organisierten Beschaffenheit des gr"o"sten Teiles der Meteorsteine ein positives und nur zu w"unschen, dass derselbe auf dem betretenen Pfade r"ustig fortschreite.
\clearpage
\section{\frakfamily{Abbildungen}}
\clearpage
\pagestyle{fancy}
\fancyhf{}
\rhead{\frakfamily{Figur 1}}
\cfoot{\thepage}
\begin{figure}[t]
\includegraphics[width=\textwidth,height=\textheight,keepaspectratio]{fig1.jpeg}
\centering
\end{figure}
\clearpage
\rhead{\frakfamily{Figur 2}}
\begin{figure}[t]
\centering
\includegraphics[width=\textwidth,height=\textheight,keepaspectratio]{fig2.jpeg}
\end{figure}
\clearpage
\end{document}
